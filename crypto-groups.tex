\documentclass[12pt]{amsart}

%\usepackage{srcltx}
\usepackage{verbatim}
\usepackage{amsmath,amsthm,amssymb,amsfonts}
\usepackage{marvosym}
\usepackage{graphicx}
\usepackage[colorlinks]{hyperref}
%\usepackage{wrapfig}
\usepackage{array}
\usepackage{amscd}
\usepackage[all]{xy}

\setlength{\extrarowheight}{1pt}

% it would be better if the bf and it did not apply in math mode
\newcommand{\terminology}[1]{\textbf{\textit{#1}}}
\renewcommand{\terminology}[1]{#1}
\newcommand{\term}{\terminology}

\newcommand{\nc}{\newcommand}
%\nc{\con}{\text{\reflectbox{\Lightning}\Lightning}}
%\nc{\con}{*}
%\nc{\con}{\includegraphics[keepaspectratio=true, width=1em]{lightn.png}}
\def\con{&nbsp;&#9889;&#9889;&nbsp;}
\nc{\Q}{\mathbb{Q}}
\nc{\Z}{\mathbb{Z}}
\nc{\C}{\mathbb{C}}
\nc{\R}{\mathbb{R}}
\nc{\F}{\mathbb{F}}
\nc{\Qstar}{\Q^\times}
\nc{\Cstar}{\C^\times}
\nc{\Rstar}{\R^\times}
\nc{\Znstar}{\Zn^\times}
\nc{\Nat}{\mathbb{N}}
\nc{\Zp}{\Z_p}
\nc{\Zn}{\Z_n}
\nc{\Fp}{\F_p}
\nc{\pr}{\mathbb{P}}
\nc{\af}{\mathbb{A}}
\nc{\bx}{\mathbf{x}}
\nc{\Fpbar}{\overline{\F}_p}
\nc{\Kbar}{\overline{K}}
\nc{\Qbar}{\overline{\Q}}
\nc{\Qpbar}{\overline{\Q}_p}
\nc{\cfp}{\mathcal{F}_p}
\nc{\calA}{\mathcal{A}}
\nc{\calC}{\mathcal{C}}
\nc{\calP}{\mathcal{P}}
\nc{\calK}{\mathcal{K}}
\nc{\Oc}{\mathcal{O}}
\nc{\orig}{\Oc}
\nc{\fal}{f_\alpha}
\nc{\cots}{, \ldots,}
\nc{\ccots}{: \cdots:}
\nc{\seq}{\subseteq}
\nc{\pots}{+ \cdots +}
\nc{\leg}[2]{{\displaystyle\left(\frac{#1}{#2}\right)}}
\nc{\ty}[1]{\texttt{#1}}
\nc{\of}{\circ}
\nc{\ds}{\displaystyle}
\nc{\exdiv}{\,||\,}
\nc{\Nr}{\textrm{N}}
\nc{\gen}[1]{\langle #1\rangle}
\nc{\frob}{\mathcal{F}}
\nc{\frobp}{\frob_p}
\nc{\frobq}{\frob_q}
\renewcommand{\Im}{\mathrm{Im}} % don't need imaginary part
\DeclareMathOperator{\Div}{Div}
\DeclareMathOperator{\Pic}{Pic}
\DeclareMathOperator{\ord}{ord}
\DeclareMathOperator{\ddiv}{div}
\DeclareMathOperator{\cok}{cok}
\DeclareMathOperator{\disc}{disc}
\DeclareMathOperator{\Hom}{Hom}
\DeclareMathOperator{\End}{End}
\DeclareMathOperator{\im}{Im}
\DeclareMathOperator{\GL}{GL}
\DeclareMathOperator{\chr}{char}


\nc{\cbar}{\overline{E}}
\nc{\abar}{\overline{a}}
\nc{\bbar}{\overline{b}}

\newcounter{probs}

\newenvironment{prob}{%
  \refstepcounter{probs}
  \par\medskip\noindent\textbf{Exercise \theprobs:} }{\par\medskip}
\nc{\pp}{\pr^1}
\nc{\tg}{\langle e\rangle}
\DeclareMathOperator{\tr}{Tr}
\DeclareMathOperator{\N}{N}
\DeclareMathOperator{\charr}{char}
\DeclareMathOperator{\aut}{Aut}
\DeclareMathOperator{\gal}{Gal}
\nc{\uph}{\mathcal{H}}
\nc{\from}{\leftarrow}
\nc{\ilim}{\displaystyle\lim_{\from}}

\theoremstyle{plain}
\newtheorem{thm}{Theorem}[section]
\newtheorem{prop}[thm]{Proposition}
\newtheorem{lemma}[thm]{Lemma}
\newtheorem{cor}[thm]{Corollary}

\theoremstyle{definition}
\newtheorem{defn}[thm]{Definition}

\theoremstyle{remark}
\newtheorem*{notation}{Notation}
\newtheorem*{note}{Note}
\newtheorem*{remark}{Remark}
\newtheorem*{exam}{Example}

%\let\oldsec\section
%\def\section{\clearpage\oldsec}

\title{Algebra for Cryptologists}
\author{John W.{} Jones}

\begin{document}
\maketitle

These notes were written for the beginning of MAT 448, \emph{Cryptography
II}.  There are basic notions from abstract algebra, particularly from
group theory, which are essential
throughout the course.  On the other hand, we will not need advanced
aspects of group theory like the Sylow theorems, so we will cover
everything we need relatively quickly.


\tableofcontents

\section{Notation and conventions}
Note that in proofs by contradiction, we use the symbol \con{} to
indicate that a contradiction has been reached.

\subsection{Famous sets}
Here we collect definitions and notation to be used throughout.  It is
intended more as a reference than an introduction.  In particular, we
specify what natural algebraic structures these sets possess even
though those structures are not defined until later sections.

We let
\begin{enumerate}
\item[] $\term{\Z}$ :  the set of integers \quad (a group under $+$ and a ring)
\item[] $\term{\Q}$ :  the set of rational numbers  \quad (a group under $+$ and a field)
\item[] $\term{\R}$ :  the set of real numbers  \quad (a group under $+$ and a field)
\item[] $\term{\C}$ : the set of complex numbers \quad (a group under $+$ and a field)
\item[] $\term{\Qstar}$ : $\{a\in\Q\mid a\neq 0\}$ \quad (a group under multiplication)
\item[] $\term{\Rstar}$ : $\{a\in\R\mid a\neq 0\}$ \quad (a group under multiplication)
\item[] $\term{\Cstar}$ : $\{a\in\C\mid a\neq 0\}$\quad (a group under multiplication)
\item[] $\term{\Zn}$ :  integers modulo $n$ \quad (a group under $+$ and a ring)
\item[] $\term{\mathbb N}$ : $\{a\in\Z\mid a\geq 0\}$ 
\end{enumerate}
%\begin{align*}
%&\Z = \text{the set of integers} \quad\text{(a group under $+$ and a ring)} \\
%&\Q = \text{the set of rational numbers}  \quad\text{(a group under $+$ and a field)} \\
%&\R = \text{the set of real numbers}  \quad\text{(a group under $+$ and a field)} \\
%&\C= \text{the set of complex numbers} \quad\text{(a group under $+$ and a field)} \\
%&\Qstar= \{a\in\Q\mid a\neq 0\} \quad\text{(a group under multiplication)}\\
%&\Rstar= \{a\in\R\mid a\neq 0\} \quad\text{(a group under multiplication)}\\
%&\Cstar= \{a\in\C\mid a\neq 0\}\quad\text{(a group under multiplication)}\\
%&\Zn = \text{integers modulo $n$} \quad\text{(a group under $+$ and a ring)} \\
%&\mathbb N=\{a\in\Z\mid a\geq 0\} 
%\end{align*}
A superscript of $+$ means we want the positive elements, so
$$\term{\Z^+}=\{a\in \Z\mid a>0\},$$ and similarly for $\term{\Q^+}$ and $\term{\R^+}$.

If we have a set $R$ and positive integer $n\in\Z^+$, then we let
\[ M_n(R)=\{n\times n \text{ matrices with entries from $R$}\}.\]
We will only use this when $R$ is a ring, specifically $R=\Q$, $\R$,
$\C$, $\Zn$, or $\Z$.  
% In any of these cases, $M_n(R)$ is a ring with
% identity under matrix addition and matrix multiplication.  We let
% $I_n\in M_n(R)$ be the identity matrix (ones on the main diagonal and
% zeros everywhere else).
% Since $M_n(R)$ is a ring with identity, we can consider its group of
% units:
In each case, we are also interested in the group
\[ \GL_n(R) =\{A\in M_n(R)\mid \det(A)\in R^\times\}.\] 
For this definition, $\Zn^\times$ is defined in Section~\ref{zn}
below; $\Z^\times=\{-1,1\}$.

\subsection{Functions}
If we have a function $f:A\to B$, then we assume $f$ is defined for
every element of $A$, so $A$ is the domain of $f$.  We refer to $B$ as
the codomain of $f$, as opposed to its image:
\[ \text{Im}(f)=f(A) = \{f(a)\mid a\in A\}\seq B\]

\subsection{$\Zn$}
\label{zn}
An essential construction for us is $\Zn$.  There are three ways of
viewing it.  First, we fix $n\in\Z^+$.

First, we can think of integers up to congruences.  If $a,b\in \Z$, we
define
\[ a\equiv b\pmod n \iff n\mid b-a.\]
Here, we are dealing with integers, but only think of them as
different if they are not congruent modulo $n$.  Every integer is
congruent to its remainder on division by $n$, and no two remainders
from $\{0,\ldots, n-1\}$ are congruent modulo $n$, so we have $n$
things (see Section~\ref{div-alg} for more on remainders and the
Division algorithm).

Building on this idea gives the second point of view, namely that
$\Zn=\{0\cots n-1\}$, and when we perform operations, we always
replace the result with its remainder on division by $n$.  This is
very concrete and well-suited for implimentation on a computer, but is
less elegant mathematically.  To assist in working with this version,
we define
\[ a\bmod n = \text{ the remainder when $a$ is divided by $n$}.\]
This is the unique integer $r$ such that $0\leq r<n$ and $a=nq+r$ for
some integer $q$.

The third approach also builds on the first.  One can show that
congruence modulo $n$ is an equivalence relation on $\Z$, in which
case we have equivalence classes:
\[ [a]_n = \{b\in\Z\mid a\equiv b\pmod n\}.\]
From the comments above, we again get a set with $n$ things, namely
$\{[0]_n, [1]_n, \ldots, [n-1]_n\}$, but with the classes, $[0]_n=[n]_n$, and
$[5]_n=[n+5]_n=[-n+5]_n$.

We can sum up the connections between these points of view as
follows.  If $n\in\Z^+$ and $a,b\in \Z$
\[
\begin{array}{ccccc}
\text{Classes} && \text{Relation} && \text{Remainders} \\[1ex]
[a]_n=[b]_n & \iff & a\equiv b\pmod n & \iff & (a\bmod n) = (b \bmod n)
\end{array}
\]

In all cases, we want to be able to add and multiply elements of
$\Zn$.  From the point of view of congruences, this says that we can
replace one integer with another it is congruent to when performing
addition or multiplication.
\begin{prop}\label{cong-well-def}
  Let $n\in\Z^+$, $a,b,c,d\in \Z$ with 
  \[ a\equiv b \pmod n \quad\text{and}\quad c\equiv d\pmod n.\]
  Then,
  \begin{enumerate}
  \item $a+c\equiv b+d\pmod n$
  \item $ac\equiv bd \pmod n$
  \item $a-c\equiv b-d\pmod n$
  \end{enumerate}
\end{prop}
We leave the proofs as exercises to the reader.  Note, the final part
can be deduced from the first two since $a-c=a+(-1)\cdot c$.
\begin{prob}
  Prove Proposition~\ref{cong-well-def}.
\end{prob}


The corresponding statement in terms of congruence classes looks like
this:
\begin{cor}
  Let $n\in\Z^+$, $a,b,c,d\in \Z$ with 
  \[ [a]_n=[b]_n  \quad\text{and}\quad [c]_n= [d]_n.\]
  Then
  \begin{enumerate}
  \item $[a+c]_n=[b+d]_n$
  \item $[ac]_n= [bd]_n$
  \item $[a-c]_n=[b-d]_n$
  \end{enumerate}
\end{cor}
Phrased in this way, it shows that addition, multiplication, and
subtraction of congruence classes is well-defined.  That is, the
operations can be computed by chosing elements from the classes, and
the final result does not depend on the choices.


Using $\Zn=\{0\cots n-1\}$, we can write down complete addition and
multiplication tables for small $n$.  For example, here are the
operation tables for $\Z_4$.
\[
\begin{array}{c|*{4}{|c}}
+ & 0 & 1 & 2 & 3 \\ \hline \hline
0 & 0 & 1 & 2 & 3 \\ \hline
1 & 1 & 2 & 3 & 0 \\ \hline
2 & 2 & 3 & 0 & 1 \\ \hline
3 & 3 & 0 & 1 & 2 
\end{array}
\qquad
\qquad
\begin{array}{c|*{4}{|c}}
\cdot & 0 & 1 & 2 & 3 \\ \hline \hline
0 & 0 & 0 & 0 & 0 \\ \hline
1 & 0 & 1 & 2 & 3 \\ \hline
2 & 0 & 2 & 0 & 2 \\ \hline
3 & 0 & 3 & 2 & 1 
\end{array}
\]

We consider another instance of something being \emph{well-defined}.
If $[a]_n\in\Zn$, then we would like to define $\gcd([a]_n,n)$ to be
simply $\gcd(a,n)$.  Since it is possible for a single congruence
class to be represented by many different integers, we have to prove
that the final result does not depend on this choice.
\begin{prop}
  If $n\in\Z^+$ and $a,b\in\Z$ with $[a]_n=[b]_n$, then
  $\gcd(a,n)=\gcd(b,n)$.
\end{prop}
\begin{proof} \label{gcd-well-def}
  Since $[a]_n=[b]_n$, $a\equiv b\pmod n$, which implies that 
\begin{equation}\label{abnk}
  a=b+nk
\end{equation}
for some $k\in\Z$.  Since $\gcd(b,n)\mid b$ and $\gcd(b,n)\mid n$,
  we get from equation~\ref{abnk} that $\gcd(b,n)\mid a$.  Since
  $\gcd(b,n)\mid n$ as well, we get $\gcd(b,n)\leq \gcd(a,n)$.

  We can also solve equation~\ref{abnk} for $b$: $b=a+n(-k)$.
  Repeating the same argument with the roles of $a$ and $b$ reversed
  gives $\gcd(a,n)\leq \gcd(b,n)$.  Together with the first part, we
  get $\gcd(a,n)=\gcd(b,n)$.
\end{proof}
\begin{exam}
  To see an example where something is \emph{not} well-defined,
  supposed we wanted to define $\gcd([a]_n,[b]_n)$ to be $\gcd(a,b)$.
  This will not work in general; here is a counterexample.  Let $n=3$,
  $a=b=1$.  Then $\gcd(a,b)=\gcd(1,1)=1$.  But, $[1]_3=[4]_3$.  On one
  hand, we would have $\gcd([1]_3,[1]_3)=1$, and on the other hand
  $\gcd([1]_3,[1]_3) = \gcd([4]_3,[4]_3) =\gcd(4,4)=4$.  The result
  depends on what integers we pick for representing the congruence classes.
\end{exam}

By Proposition~\ref{gcd-well-def}, we can define
\[ \term{\Znstar} = \{[a]_n\in\Zn\mid \gcd(a,n)=1\},\]
and define the Euler phi-function for $n\in\Z^+$:
\[ \term{\varphi(}n) = |\Znstar|.\]
Note, although $\Z_1=[0]_1$ has only one element, $\gcd(0,1)=1$ so
$\Z_1^\times=\{[0_1]\} = \{[1]_1\}$, giving $\varphi(1)=1$.




\section{Groups}

Here we introduce definitions and some notions from group theory.  As
we will see, they apply to crytography in many situations and can
unify ideas applied in different places.

We start with something more basic, the notion of a binary operation
which is central to the definition of group.

\subsection{Binary operations}

\begin{defn}
A \term{binary operation} $*$ on a set $S$ is a function
\[ *:S\times S \to S.\]
We write $a*b$ for the value of this function on the ordered pair
$(a,b)\in S$.
\end{defn}
Note:
\begin{itemize}
\item by saying that $*$ is a function, it implies that $a*b$ is defined
  for every pair of elements $a,b\in S$.
\item The codomain is $S$, so always $a*b\in S$.
\end{itemize}
\begin{exam}
Three familiar examples are given by addition, subtraction, and
multiplication on $\Z$.
\end{exam}

\begin{exam}
Similarly, addition, subtraction,
and multiplication are each binary operations on $\R$, but division is
not a binary operation since $5\div 0$ is not defined as a real number.
\end{exam}

\begin{exam} On the other hand, division is a binary operation on
$\R^\times$.
\end{exam}

There are several properties of interest which a binary operation may
satisfy.
\begin{defn}
  Let $*$ be a binary operation on a set $S$.
\begin{itemize}
\item  $*$ is \term{associative} if for all
  $a,b,c\in S$,
  \[ a*(b*c)=(a*b)*c.\]
\item $*$ is \term{commutative} if for all $a,b\in S$,
  \[ a*b = b*a.\]
\item $*$ has \term{identity} $e$ if $e\in S$ and for all $a\in S$,
  \[ a*e=e*a=a.\]
\end{itemize} 
\end{defn}

\begin{exam}
Addition and multiplication on $\Z$ are both commutative,
associative, and have identity ($0$ for addition and $1$ for
multiplication).  Note, subtraction is neither commutative ($3-5\neq
5-3$) nor associative ($1-(2-3)\neq (1-2)-3$), and does not have an
identity element.
\end{exam}

\begin{prob}
Prove that $\Z$ does not have an identity element under subtraction.
\end{prob}

Our first proposition shows that if there is an identity element for a
binary operation, then it is unique.
\begin{prop} \label{unique-identity}
  If $*$ is a binary operation on a set $S$, then there is at most one
  identity element for $*$.
\end{prop}
\begin{proof}
  Suppose to contrary that $e$ and $e'$ are both identity elements
  for $*$ on $S$.  Then
  \begin{align*}
    e*e' &= e \text{ since $e'$ is an identity element, and} \\
    e*e' &= e' \text{ since $e$ is an identity element.}
  \end{align*}
  Thus $e=e'$.
\end{proof}
We have one more definition.
\begin{defn}
  Let $*$ be a binary operation on a set $S$ which has identity $e$.
  If $a\in S$, we say that $b\in S$ is an \term{inverse} of $a$ if
  \[ a*b=b*a=e.\]
\end{defn}
When an inverse exists, it is unique.
\begin{prop} \label{unique-inverse}
  Let $*$ be an associative binary operation on a set $S$ which has
  identity $e$, and $a\in S$.  Then $a$ has at most one inverse in
  $S$.
\end{prop}
\begin{proof}
  Suppose $b$ and $c$ are both inverses of $a$.  Then
\[ b*(a*c) = b*e= b\]
and
\[ (b*a)*c = e*c=c.\]
Since the operation is associative, $b*(a*c)=(b*a)*c$, so $b=c$.
\end{proof}

\begin{exam}
In $\Z$ under addition where $0$ is the identity, every integer
$n\in\Z$ has an inverse $-n$ because $n+(-n) = 0=(-n)+n$.  On the
other hand, for $\Z$ under multiplication (where $1$ is the identity),
the only elements with inverses are $1$ and $-1$ (because $1\cdot 1 =
1$ and $(-1)\cdot (-1) = 1$).
\end{exam}

\begin{notation}
  In almost every binary operation we encounter, it will already be
  naturally an addition or a multiplication.  Addition in every case
  will be commutative, and we will only use $+$ for binary operations
  which are commutative.  In these cases, the inverse of $a$ will be
  denoted by $-a$.

  In other cases, we will generally use $\cdot$ instead of $*$ for the
  operation, or use no symbol at all for the operation and just write
  $ab$ for $a*b$.  In these cases, we are not assuming the operation
  is commutative unless it is explicitly mentioned.  The inverse of an
  element $a$ (if it exists) is then denoted by $a^{-1}$.
\end{notation}

\subsection{Definition of group}
\begin{defn}
  A \term{group} is a set $G$ with a binary operation $*$ such that 
\begin{enumerate}
\item $*$ is associative
\item $G$ has an identity element for $*$
\item every element of $G$ has an inverse.
\end{enumerate}
\end{defn}

\begin{exam}
  Addition is a group operation on many familiar sets: $\Z$, $\Q$,
  $\R$, $\C$.  Note, $\Z^+$ is not a group for addition because it
  does not have an identity element.  Similarly, $\mathbb N$ is not a
  group for addition because it contains elements which do not have
  inverses (such as $3$, or $5$, or any  positive integer).
\end{exam}

\begin{exam}
  Multiplication is a group operation on other familiar sets:
  $\Q^\times$, $\R^\times$, $\C^\times$, and $\R^+$.
\end{exam}

% \begin{exam}
%   The one element set $G=\{0\}$ is a group under addition.  The
%   identity element is of course $0$ (it is the only option), and $0$
%   is its own inverse.

%   In fact, $\{0\}$ is a group under multiplication.  In this case, $0$
%   functions as the identity for multiplication, and since $0\cdot
%   0=0$, it is also its own inverse.

%   In fact, these two examples are the same.  The set is the same
%   $\{0\}$, and the binary operation is the same (i.e., the function
%   $G\times G\to G$ has a unique ordered pair $((0,0), 0)$ in both
%   cases).  The only difference is the notation we use for the binary
%   operation.
% \end{exam}

\begin{remark}
From Propositions \ref{unique-identity} and \ref{unique-inverse}, we
know that the identity element of a group is unique and that each
element has a unique inverse.
\end{remark}

\begin{defn}
  A group $G$ is \term{abelian} if the operation is commutative.
\end{defn}
It might seem more natural to call these groups commutative groups,
but the terminology is universally accepted in
mathematics\footnote{The term \em{abelian} comes from the name of a
  mathematician, Abel.}.  The additive and multiplicative groups given
above are all abelian.  Here is an example which is not.
\begin{exam}
  Let $G=\GL_2(\R)$, the set of $2\times 2$ matrices over the real
  numbers non-zero determinant.  This is a group under matrix
  multiplication with identity being the $2\times 2$ identity matrix
  and inverses being inverse matrices (using that for $2\times 2$
  matrices over $\R$, a matrix has an inverse if and only if it
  has non-zero determinant).  However, this group
  is not commutative.
\end{exam}

\section{Subgroups}

We often have groups which are subsets of bigger groups.  Suppose $G$
is a group and $H\seq G$.
If we restrict the domain of the binary operation from
$G\times G$ to $H\times H$, and then want the results to always lie in
$H$.  In other words, for all $a,b$,
\[ a,b\in H \implies a*b\in H.\]
When this happens, we say that $H$ is \term{closed under} $*$.

\begin{defn}
  If $G$ is a group with operation $*$ and $H\seq G$, then we say that
  $H$ is a \term{subgroup} of $G$ if
  \begin{enumerate}
  \item $H$ is closed under $*$,
  \item the identity element $e$ for $G$ is in $H$,
  \item for all $a\in H$, its inverse (in $G$) is contained in $H$.
  \end{enumerate}
\end{defn}

One could say that a subset is a subgroup if it is closed under the
binary operation, contains the identity, and is closed under
inverses.

Our definition is not an literal match to the concept (subset which is
also a subgroup).  The missing steps are explored in the exercises.

\begin{exam}
  Addition is a binary operation on $\R$.  If we restrict to rational
  numbers $a,b\in\Q$, then $a+b\in\Q$, so $\Q$ is closed for $+$.
\end{exam}
\begin{exam}
  On the other hand, division is a binary operation on $\Q^+$, the set
  positive rational numbers.  However, if we try to restrict to
  positive integers $a,b\in\Z^+$, then $a\div b$ is a positive
  rational number, but it need not be a positive integer.  For
  example, $3\div 7\not\in\Z^+$.  So $\Z^+$ is not closed under
  division.
\end{exam}

The first problem makes it easier to prove that an element of a group
is the identity element.
\begin{prob} \label{ident-prob}
Let $G$ be a group with identity $e$.  Then $c*c=c$ if and only if $c=e$.
\end{prob}

\begin{prob}
  We would like to define subgroup as follows: ``If $G$ is a group and
  $H\seq G$, then $H$ is a \emph{subgroup} of $G$ if the binary
  operation for $G$ induces a binary operation on $H$ and $H$ is a
  group for that operation.''  A subset which is a subgroup under our
  definition definitely satisfies this definition as well.  To show
  the implication goes both ways, prove that if $H$ satisfies the
  quoted statement, then
  \begin{enumerate}
  \item $H$ is closed for $*$,
  \item $e_G=e_H$ (hint, use problem~\ref{ident-prob})
  \item for all $a\in H$, the inverse of $a$ in $G$ is in $H$.
  \end{enumerate}
\end{prob}


\section{Groups acting on sets}

Groups are fun, but they are much more fun when they are doing
something, like acting on a set.
\begin{defn}
  Let $G$ be a group and $S$ a set.  An \term{action} of $G$ on $S$ is a
  function
  \[ \cdot:G\times S\to S\]
  such that
  \begin{enumerate}
  \item for all $s\in S$, $e\cdot s = s$ where $e$ is the identity of
    $G$;
  \item for all $g_1, g_2\in G$ and all $s\in S$, $g_1\cdot (g_2\cdot
    s) = (g_1g_2)\cdot s$.
  \end{enumerate}
\end{defn}
In the second condition, both dots on the left hand side represent the
action where as $g_1g_2$ on the right hand side is multiplication
within the group.

We will see several examples of group actions in the next section
where they have applications to cryptography, so we just give two
quick ones now.

\begin{exam}
The group $G=\Z$ acts on $S=\R$ by $n\cdot a=n+a$.  In other words,
we use addition to define the action.
\end{exam}

\begin{prob}
  Prove that the action of $\Z$ on $\R$ given by $n\cdot a=n+a$ is a
  group action.
\end{prob}

\begin{exam} If $G$ is a group, define an action of $G$ on $G$ by $g\cdot
a=ga$.  In other words, we use the group operation to define the
action.  This particular action is famous and is known as \term{left
  translation}.
\end{exam}

\begin{prob}
  Prove that if $G$ is a group, then left translation gives a
  group action.
\end{prob}


\section{Cryptosystems from group actions}
For background and notation for cryptosystems, see Section~\ref{crypto-101}.


Suppose $G$ is a group which acts on a set $A$, and we find a
way to match elements of $A$ with the plaintexts.  Then we will let
$A$ also be the set of ciphertexts, and $G$ is the set of keys.  Given
a key $k\in G$ and a plaintext $P\in A$, our encryption function is
\[ e_k(P) = k\cdot P\]
where the right hand side comes from the group action.  The decryption
function uses the inverse of $k$ in the group:
\[ d_k(C) = k^{-1}\cdot P\]
To check that this works, we compute:
\begin{align*}
d_k(e_k(P)) &= d_k(k\cdot P) \\
  &= k^{-1}\cdot (k\cdot P) \\
  &= (k^{-1}k)\cdot P & \text{property 2 of group action} \\
  &= e\cdot P \\
  &= P & \text{property 1 of group action}
\end{align*}
We now consider some classic ciphers, and see how they are examples of
this one idea.
\subsection{Shift Cipher}
For the \term{shift cipher},
$\calP=\calC=\calA=\calK=\Zn=G$.
We let $G$ act on itself by left
translation, so for a key $k\in G$, our encryption is given by
\[ e_k(p) = p+k\]
where the addition is taken modulo $n$ since we are working in $\Zn$.
Decryption is simply
\[ d_k(c) = c-k.\]


\subsection{Affine Cipher}
For the \term{affine cipher}, we let $\calA=\calP=\calC=\Zn$.
Our group is
the \term{affine group modulo $n$}:
\[ \textit{Aff}(n) = \{ ax+b\mid a\in \Zn^\times\text{ and } b\in\Zn\}\]
where the operation is composition.
\begin{prob}
  Verify that if $n\in\Z^+$, then $\textit{Aff}(n)$ is a group.
\end{prob}
If $ax+b\in \calK= \textit{Aff}(n)$, then $ax+b$ acts on $m\in \Zn$ by
\[ (ax+b)\cdot m = am+b\]
with the computation done modulo $n$.  This gives encryption functions
\[ e_{ax+b}(p) = ap+b.\]
Since $a\in \Z_n^\times$,
there exists $a'\in\Z_n^\times$ such that $aa'\equiv 1\pmod n$.  Then
\[d_{ax+b}(c) = a'(c-b).\]

\subsection{Hill Cipher}
For the \term{Hill cipher}, we let $\calA=\Zn$ and pick
$m\in\Z^+$.  Then we take $\calP=\calC=\Z_n^m$.
The group 
\[ G=\GL_m(\Zn)=\{A\in M_m(\Zn)\mid \det(A)\in \Z_n^\times\}\]
is the group of matrices over $\Zn$ which are invertible for matrix
multplication.  It acts on $\Z_n^m$ in the usual way of matrix times
vector.

If $A\in \calK=\GL_m(\Zn)$ and $\vec p \in \Z_n^m$, then encryption is
\[ e_A(\vec p) = A\vec p\]
and decryption is
\[ d_A(\vec c) = A^{-1}\vec c,\]
where $A^{-1}$ is the inverse of $A$ modulo $n$.


\subsection{Permutation Cipher}
Let $n\in\Z^+$.  A \term{permutation} on a set $A$ is a bijective
function $f:A\to A$.  The \term{symmetric group} on $A$ is
\[ S_A = \{ f:A\to A\mid \text{$f$ is a permutation}\}.\]
This is a group under composition:
\begin{itemize}
\item the composition of bijective functions is bijective
\item function composition is always associative
\item the identity function $I_A:A\to A$ given by $I_A(a)=a$ for all
  $a\in A$ acts as the identity under composition
\item bijective functions have inverse functions (which are also bijective)
\end{itemize}
For any set $A$, $S_A$ acts on $A$ by $f\cdot a=f(a)$ for any $f\in S_A$ and any
$a\in A$.

In the special case of $A=\{1\cots n\}$, we write $S_n$ for the set of
permutations.  Then $S_n$ acts on $n$ tuples from another set $B$ by
\[ \sigma \cdot (b_1\cots b_m) = (b_{\sigma(1)}\cots b_{\sigma(n)}).\]

So, we fix $n$ and take $\calA^n=\calP=\calC$ and $\calK=S_{n}$.  If
$\sigma\in S_n$, encoding is given by
\[ e_\sigma(a_1\cots a_n) = (a_{\sigma(1)}\cots a_{\sigma(n)})\]
and decoding by
\[ d_\sigma(a_1\cots a_n) = (a_{\sigma^{-1}(1)}\cots a_{\sigma^{-1}(n)}).\]

\subsection{Substitution Cipher}


If $\calA$ is our alphabet, we let $\calP=\calC=\calA$ and key space
$\calK=S_\calA$. 
Then for any $f\in S_\calA$, we encrypt and decrypt by
\[ e_f(p) = f(p) \quad\text{ and }\quad d_f(c) = f^{-1}(c).\]

Many other ciphers are special cases of the substitution cipher, such
as shift, affine, and RSA.  If $n=|\calA|$, then the size of the key
space is $n!$, which grows very quickly.  This means that to
communicate a key with large $n$, one must transmit many bits.  The
more specialized ciphers have smaller key spaces, and hence,
transmitting the key is easier.  If $\calA$ is an ordinary alphabet
from a natural language, then this cipher can be readily attacked with
frequency analysis.

\subsection{Vigen\`ere Cipher}
\label{vigenere}
We first introduce a way to construct new groups from others which
we will need later.

If $G_1$ and $G_2$ are groups with operations $*_1$ and $*_2$, then
we can make $G_1\times G_2$ into a group with the operation
\[ (a_1,a_2)*(b_1,b_2) = (a_1*_1 b_2, a_2*_2 b_2) \]
It has identity $(e_1, e_2)$ and the inverse of an element $(a_1,a_2)$
is $(a_1^{-1}, a_2^{-2})$ where the inverses are computed in $G_1$
and $G_2$ respectively.  The resulting group is called the {\em
direct product} of $G_1$ and $G_2$, and is denoted $G_1\times G_2$.
In some cases, it may also be denoted by $G_1\oplus G_2$.

This construction can be iterated with a list of groups
$G_1\cots G_n$.  The elements then are $n$-tuples where the $i$th
coordinate comes from $G_i$.

For a Vigen\`ere cipher, we take $m$ to be a positive integer, and let
$\calP=\calC=\calA^m$.  Let $n=|\calA|$ and identify $\calA$ with
$\Zn$.  Our group is $G=\Zn\times\Zn\times\cdots \times \Zn$ where we
use $m$ copies of $\Zn$.  So a key is an $m$-tuple of elements of
$\calA$, i.e., an $m$-letter word.  The group action is $G$ acting on
itself by left translation.

\subsection{RSA}
The main feature of RSA is that it is a public key cryptosystem.  But
under the hood, it works on the same principle of groups acting on
sets.

Let $N$ be a positive integer which is a product of distinct primes.
We let $\calP=\calC=\Z_N$.  The traditional choice for the group is
$G=\calK=\Z_{\varphi(N)}^\times$.  It acts on $\Z_N$ as follows
\begin{align*}
\Z_{\varphi(N)}^\times\times\Z_N &\to \Z_N   \\
(a, b) &\mapsto b^a
\end{align*}
For encryption, we pick a key $a$ such that $\gcd(a, \varphi(N))=1$,
i.e., such that $a\in\Z_{\varphi(N)}$ and then
\[ e_a(m) = m^a \bmod N.\]
For decryption, we find $b$ such that $ab\equiv 1\pmod{\varphi(N)}$,
but this is just $a^{-1}\in\Z_{\varphi(N)}^\times$ and
\[ d_a(c) = c^b\bmod N.\]

It may be hard to see why we take the hypothesis that the prime
factors of $N$ must be distinct.  We illustrate the problem with an
example.
\begin{exam}
  Let $N=9=3^2$, so $\varphi(N) = 6$.  Then $1\equiv 7
  \pmod{\varphi(N)}$ and certainly $\gcd(1,6)=1$, so $1\in
  \Z_6^\times$.  If we compute the group action with $1$ and
  $3\in\Z_9$, we get $3^1=3$, but if we use $7$, we get $3^7\equiv
  0\pmod 9$, a different result.  This is not a group action because
  the ``action'' needs to be a function, and it is not well-defined.
\end{exam} 
\begin{prob}
  Suppose $N$ is a product of distinct primes and $m=\varphi(N)$.
  Prove that if $[a]_m=[b]_m\in\Z_m$ and $[c]_N\in\Z_N$, then $[c^a]_N=[c^b]_N$. 
\end{prob}

\subsection{LFSRs}
For background on LFSRs, see Chapter~\ref{lfsr}.  

If an $n$-stage LFSR has associated matrix $C\in\GL_n(\Z_2)$, then
groups enter directly since the order of $C$ in the group
$\GL_n(\Z_2)$ gives the longest period one can achieve from the LFSR.

In general, if a group $G$ acts on a set $S$, we can make a stream
cipher with one more ingredient, a function $f:S\to\{0,1\}$.  Then
we choose an element $g\in G$ and an initial
element $s_0\in S$.  We construct a sequence of elements of $S$, and
the corresponding stream of $x_i \in \{0,1\}$ via
\[ s_n = g\cdot s_{n-1} \qquad x_n = f(s_n).\]

In the case of an LFSR, the group is $\GL_n(\Z_2)$, the set
$S=\Z_2^n$, and the action is the usual matrix times vector giving
vector.  Then $g=C$, the matrix associated to the LFSR, and
$s_0\in\Z_2^n$ is the initial load of the LFSR.  The function $f$ is
given by $f(a_1,\ldots, a_n) = a_n$.

\subsection{Pseudo-random number generators}
It is worth noting that a small generalization of this construction
produces another important object in cryptography.  Many aspects of
cryptography make use of random number numbers.  Often times, these
need not be truely random, and a computer language's random number
generator is good enough.  These are not random; they are completely
determined by the algorithm used and an initial seed.

A commonly used pseudo-random number generator is a {\em linear
  congruenential generator}.  Suppose you want pseudo-random numbers
in $\Z_{2^k}$.  Then pick a large positive integer $m>2^k$, $a\in\Z_m^\times$, and
$b\in\Z_m$.  Starting with an initial seed $s_0\in \Z_m$, we construct
the sequence defined recursively by $s_n = as_{n-1}+b \bmod m$.  The
random numbers are then given by something like taking the top
$k$-bits of $s_n$, i.e., it uses a function $f:\Z_m\to \Z_{2^k}$ which
extracts $k$ bits.

This fits our setup for a stream cipher from a group action where 
\[G=\textit{Aff}(m) = \{ax+b\mid a\in\Z_m^\times\text{ and }
  b\in\Z_m\}\]
and $S=\Z_m$ (the action given by plugging a value into the linear
polynomial).  The only difference is the codomain of the function
$f$.  In a stream cipher, it would be a function $f:S\to \Z_2$; for
the random number generator, it is $f:S\to \Z_{2^k}$.


\section{Isomorphisms and homomorphisms}

Consider the following two operation tables for $\{1,-1\}$ under
multiplication and $\Z_2$:
\[
\begin{array}{c||c|c}
\cdot & 1 & -1 \\ \hline\hline
1 & 1 & -1 \\ \hline
-1 & -1 & 1
\end{array}
\qquad
\begin{array}{c||c|c}
+ & 0 & 1 \\\hline\hline
0 & 0 & 1 \\ \hline
1 & 1 & 0
\end{array}
\]
Then we can get from the table on the left to the table on
the right by simply renaming elements systematically.  In
particular, if we take the table on the left and use the
following replacements:
\begin{align*}
1 &\mapsto 0\\
-1 &\mapsto 1
\end{align*}
This is the idea of an isomorphism.  Renaming elements from one set
to elements of another is formalized as a bijective function between
the sets $f:G_1\to G_2$.  
The idea that the group tables match up can be thought of elements
$a\in G_1$ have their old name, $a\in G_1$, and a new name 
$f(a)\in G_2$.  Then the group tables matching after renaming amounts
to, 
\begin{quote}
if we take any two elements of $G_1$, multiply in $G_1$, and then rename
the result, it is the same as first renaming the elements and then
multiplying them in $G_2$.
\end{quote}
Formally, this leads to
\begin{defn}
  If $G_1$ is a group with operation $*$ and $G_2$ is a group with
  operation $*'$, then an \term{isomorphism} from $G_1$ to $G_2$ is a
  bijective function
  \[ f:G_1\to G_2\]
  such that
  \[ f(a*b)  = f(a)*'f(b)\qquad\text{for all $a,b\in H$.}\]
  We then say that $G_1$ and $G_2$ are \emph{isomorphic} and write
  $G_1\cong G_2$.
\end{defn}
The idea of ``same except for renaming of elements'' should behave
like an equivalence relation.  The formal proof of this is left
as a sequence of exercises.
\begin{prob}
Prove that every group is isomorphic to itself.  (Hint: for any
set $G$, there is a simple bijective function $G\to G$.)
\end{prob}
\begin{prob}
Suppose $G_1$ is isomorphic to $G_2$.  Prove that $G_2$ is isomorphic
to $G_1$.  (Hint: the hypothesis gives you a bijective function
$G_1\to G_2$; you need to start by constructing a bijective function
$G_2\to G_1$.
\end{prob}
\begin{prob}
Suppose $G_1$ is isomorphic to $G_2$ and $G_2$ is isomorphic to $G_3$.
Prove $G_1$ is isomorphic to $G_3$.
\end{prob}
Here is an example of an isomorphism from familiar objects.
\begin{prob}
Prove that the group $\R$ under addition is isomorphic to the group
$\R^+$ under multiplication. (Hint: try an exponential map.)
\end{prob}

It is useful to consider functions between groups which respect the
operations, but are not necessarily bijective:
\begin{defn}
  If $H$ is a group with operation $*$ and $K$ is a group with
  operation $*'$, then a function
  \[ f:H\to K\]
  such that
  \[ f(a*b)  = f(a)*'f(b)\qquad\text{for all $a,b\in H$}\]
  is called a \term{homomorphism} from $H$ to $K$.
\end{defn}

\begin{exam}
  The inclusion map $i:\Z\to \Q$ (defined by $i(n)=n$ for all integers
  $n$) is a homomorphism which is 1-1, but not onto.
\end{exam}
\begin{exam}
  If $n$ is a positive integer, the map $f:\Z\to \Zn$ given by 
  \[ f(a) = \text{ the congruence class of $a$ modulo $n$} \]
  is a homomorphism called the \term{reduction map}.  It is onto, but
  not 1-1.
\end{exam}
\begin{exam}
  If $H$ and $K$ are any two groups, the constant function $f:H\to K$
  given by $f(a)=e_K$ for all $a\in H$ is a homomorphism, called the
  \term{trivial map}.  If $H$ and $K$ have more than one element each,
  it is neither 1-1 nor onto.
\end{exam}

We collect the basic properties of homomorphisms in the next proposition.
\begin{prop}
If $f:H\to K$ is a group homomorphism, then
\begin{enumerate}
\item $f(e_H) = e_K$
\item for all $h\in H$, $f(h^{-1})=f(h)^{-1}$
\end{enumerate}
\end{prop}
\begin{proof}
  We start by noting that $f(e_H)=f(e_H*e_H) = f(e_H)f(e_H)$, so $f(e_H)=e_K$ by
  exercise~\ref{ident-prob}.
  
  Let $h\in H$.  We prove the second part by showing $f(h^{-1})$ does the job of an
  inverse.  So we compute
  \[ f(h)f(h^{-1}) = f(hh^{-1}) = f(e_H) = e_K\]
  and
  \[ f(h^{-1})f(h) = f(h^{-1}h) = f(e_H) = e_K\,.\]
  So, $f(h^{-1}) = f(h)^{-1}$.
\end{proof}

\begin{exam}
  In linear algebra, one considers vector spaces and linear
  transformations between them.  According to the definitions,
  every vector space is an abelian group for $+$, and every
  linear transformation is a homomorphism.
\end{exam}

An important construction for homomorphsims is its kernel.  In
the linear algebra situation, this is the same as the kernel, which
is also called the null space.
\begin{defn}
 If $f:G\to K$ is a group homomorphism, the \term{kernel} of $f$ is
 \[ \term{\ker}(f) = \{g\in G\mid f(g) = e_K\}.\]
\end{defn}
\begin{prob}
Suppose $f:G\to K$ is a homomorphism.  Prove $\ker(f)$ is a subgroup
of $G$.
\end{prob}

An important family of examples of homomorphisms applies to any
abelian group.
\begin{exam}
  Suppose $G$ is an abelian group and $m\in\Z$.  Then the {\em
    multiplication by $m$ map} in additive notation is defined by
  \begin{align*}
    [m]:G&\to G \\
     g&\mapsto mg
  \end{align*}
  If we are using multiplicative notation, $[m](g) = g^m$.  This is a
  homomorphism whenever $G$ is abelian.
\end{exam}
Our main interest is the case when $m\in\Z^+$.
\begin{prob}
  Suppose $G$ is an abelian group written multiplicatively and
  $m\in\Z^+$.  Prove that the multiplication by $m$ map is a
  homomorphism from $G$ to $G$.
\end{prob}
Note, if $m=0$, then $[0](g)=g^0=e$ for all $g\in G$.  In other words,
$[0]$ is the trivial map from $G$ to $G$.



\subsection{Applications to $\Zn$}
The multiplication by $m$ map applies to all abelian groups.  Since
all finite abelian groups are built from cyclic groups, it helps to
understand this map on $\Zn$ in detail.  Ultimately, this comes down
to statements about congruences.  We determine the kernel here, and
study the image in Section~\ref{zn2}.


\begin{prop}
  If $n\in \Z^+$, $a\in \Z$ and $d=\gcd(a,n)$, then for all $x\in \Z$,
  \[ ax\equiv 0\pmod n \iff dx\equiv 0 \pmod n.\]
\end{prop}
\begin{proof}
  $(\Rightarrow)$ 
  Suppose $ax\equiv 0\pmod n$.  By the Bezout property \ref{bezout},
  there exists $r,s\in\Z$ such that $ra+sn=d$.  Then
  \[ dx\equiv (ra+sn)x \equiv r(ax)+n(sx)\equiv r\cdot 0+0\cdot
  (sx)\equiv 0 \pmod n.\]

  $(\Leftarrow)$ 
  Suppose $dx\equiv 0\pmod n$.  Since $d=\gcd(a,n)$,
  $a=dk$ for some integer $k$.  Then
  \[ ax \equiv (dk)x\equiv k(dx)\equiv k\cdot 0\equiv 0\pmod n.\]
\end{proof}

\begin{cor} \label{znkernels}
  If $n\in\Z^+$, $m\in\Z$, and $d=\gcd(n,m)$,  then as homomorphisms
  from $\Zn$ to itself, $\ker([m]) = \ker([d])$.
\end{cor}


\section{Cyclic groups}
If $G$ is a group and $g\in G$, then we use notation familiar notation
for repeating the operation.  For example,
\begin{align*}
g^2 &= gg\\
g^3&= ggg
\end{align*}
Working backwards, we also want $g^1=g$, $g^0=e$, $g^{-1}$ to be the
inverse of $g$ (which is already our notation).


\begin{defn}
  Let $G$ be a group and $g\in G$.  Then the \term{cyclic subgroup}
    generated by $g$ is
  \[ \gen g = \{ g^n\mid n\in\Z\}.\]
\end{defn}
If the group uses additive notation, then
\[ \gen g = \{ng\mid n\in\Z\}.\]
In either case, we should justify the name.
\begin{prop}
  Let $G$ be a group and $g\in G$.  Then $\gen g$ is a subgroup of $G$.
\end{prop}
\begin{proof}
  First note that $e=g^0\in \gen g$, and if $g^n\in\gen g$, then
  $(g^n)^{-1} = g^{-n}\in \gen g$.  Finally, if $g^n,g^m\in \gen g$,
  then $g^n\cdot g^m=g^{n+m}\in \gen g$.  So, $\gen g$ is a subgroup
  of $G$.
\end{proof}

\section{Order and torsion}
Here we introduce two closely related notions for elements of a group, \emph{order}
and \emph{torsion}.

\begin{defn}
If $G$ is a group, $g\in G$, and $n\in\Z^+$, then we say that \emph{$g$ is $n$-torsion}
if $g^n=e$.
\end{defn}
In additive notation, $g$ is $n$-torsion if $n\cdot g=0$.

\begin{defn}
  Let $G$ be a group and $g\in G$.
  \begin{enumerate}
  \item If $g^n\neq e$ for all $n\in\Z^+$, we say that $g$ has \emph{infinite order}.
  \item Otherwise, the \emph{order of $g$} is the smallest positive integer $n$ such that
  $g^n=e$.
  \end{enumerate}
  In either case, we denote the order of $g$ by $|g|$.
\end{defn}

\begin{exam}
  In $\Cstar$, $i$ (a root of $x^2+1$) satisfies $i^4=1$, but
  $i$, $i^2=-1$, $i^3=-i$ are all not equal to $1$, so $i$ has order $4$.  
  As far as torsion
  goes, we have $i^4=i^8=i^{12}=1$, so $i$ is $4$-torsion, and also 
  $8$-torsion, and $12$-torsion, and so on.
\end{exam}
In this example, we can see a general phenominon, namely that an element
of a group has only one order, but it can be $m$-torsion for many values
of $m$.

\begin{exam}
For $2\in \Rstar$, $2^n\neq 1$ for any $n\in\Z^+$, so it is an element 
of infinite order.  It is not $n$-torsion for any $n$.
\end{exam}

We will see soon than an element in a finite group always has finite order, so that will
be the case in which we are most interested.

\begin{thm} \label{orders-thm}
  Let $G$ be a group and $g\in G$.
  \begin{enumerate}
  \item If $g$ has infinite order, then all $g^i$ are distinct for $i\in \Z$.
  Moreover, $\Z \cong \gen g$ by the map $i\mapsto g^i$.
  \item If $g$ has finite order $n$, then
  \begin{enumerate}
  \item $\gen g=\{g^0\cots g^{n-1}\}$ which has order $n$
  \item if $i\in\Z$, then $g^i=g^{i \bmod n}$
  \item for all $i,j\in\Z$, 
  \[ g^i=g^j \iff i\equiv j \pmod n\]
  \item for all $i\in \Z$,
    \[ g^i=e \iff n\mid i\]
  \item $\Zn\cong \gen g$ by the map $[i]_n\mapsto g^i$.
  \end{enumerate}
  \end{enumerate}
\end{thm}
\begin{proof}
  We separate the two basic cases by a different criteria, but we will
  quickly see that they divide as described in the theorem.

  First suppose $g^i$ are distinct for all $i\in\Z$.  Then clearly $g$
  has infinite order.  The map $f:\Z\to \gen g$ given by $f(i)=g^i$ is
  then a bijection, and for all $i,j\in\Z$, 
  $$f(i+j)=g^{i+j} =g^ig^j=f(i)f(j).$$
  Thus, $f$ is an isomorphism.

  On the other hand, if they are not all distinct, then $g^i=g^j$ for
  some $i<j$.  Multiplying by $g^{-i}$ we get $g^ig^{-i} = g^jg^{-i}$
  which implies $e=g^0=g^{j-i}$.  In other words,
\begin{equation} \label{power-rep}
g^i=g^j \implies g^{j-i} = e
\end{equation}
Since $i<j$, $j-i\in\Z^+$, so $g$ has finite order.  Let $n=|g|$.

If $i\in\Z$, then by the Division algorithm \ref{div-alg}, there
exists $q,r\in\Z$ such that $i=nq+r$, $0\leq r<n$.  Then
\[ g^i = g^{nq+r} = (g^n)^q g^r = e^q\cdot g^r = g^r.\]
Since $r=i\bmod n$, this gives (b) and that $\gen g=\{g^0\cots
g^{n-1}\}$.  To show that this set has order $n$, we need to show that
$g^0\cots g^{n-1}$ are distinct.

Suppose not.  Then there exist $i,j\in\Z$ such that $g^i=g^j$ and
$0\leq i<j<n$.  But by implication~\ref{power-rep} above, this implies
$g^{j-i}=e$.  The inequalities on $i$ and $j$ imply $0< j-i < n$,
contradicting that $n$ is the order of $g$.  This completes the proof
of (a).

For part (c), note
\begin{align*}
 g^i=g^j &\iff g^{i\bmod n} = g^{j\bmod n} \\
&\iff i\bmod n=j\bmod n &\text{since these powers are distinct} \\
&\iff i\equiv j \pmod n
\end{align*}

Part (d) follows from (c) letting $j=0$.

Finally, for part (e) we note that the given map, $f:\Zn\to\gen g$
given by $f([i]_n) = g^i$ is a bijection from
(c) and the fact that $[i]_n=[j]_n$ iff $i\equiv j\pmod n$.  To see
that it is a homomorphism, we check
\[ f([i]_n+[j]_n) = f([i+j]_n) = g^{i+j}=g^ig^j=f([i]_n)f([j]_n)\,.\]
\end{proof}
\begin{cor}
  If $G$ is a group and $g\in G$, then the order of $g$ equals the
  order of the cyclic subgroup $\gen g$.
\end{cor}

Theorem~\ref{orders-thm} establishes the connection between the
order of an element and for which $m$ it is $m$-torsion.
\begin{cor}
  If $G$ is a group and $g\in G$ has order $n$, then $g$ is
  $m$-torsion iff $n \mid m$.
\end{cor}
One of the reasons $m$-torsion elements are useful to work with is
that they have an algebraic structure when $G$ is abelian.
\begin{defn}
  Let $G$ be an abelian group and $m\in\Z^+$.  We let
  \[ G[m] = \{g\in G\mid g^m=e\}.\]
  This is refered to as the \emph{$m$-torsion subgroup} of $G$.
\end{defn}
Before going any farther, we should justify the terminology.
\begin{prob}
  Prove that if $G$ is an abelian group, then for every $m\in\Z^+$,
  $G[m]$ is a subgroup of $G$.
\end{prob}
We note that for non-abelian groups, the corresponding sets do not
have to be subgroups.

The $m$-torsion subgroup contains all of the $m$-torsion elements.
Note, the analogous construction for elements of order $m$ is not a
subgroup (unless $m=1$), since the identity element always has order
$1$ and a subgroup must contain the identity.


\subsection{Back to $\Zn$}\label{zn2}
Here we analyze orders of elements in $\Zn$.
\begin{prop} \label{imageszn}
  Let $n\in\Z^+$, $a\in \Zn$, and $d=\gcd(a,n)$.  Then
  \[ \gen a = \gen d.\]
\end{prop}
\begin{proof}
  Since $d\mid a$, $a=dk$ for some $k\in \Z$.  Then $a\in \gen d$, so
  $\gen a = \gen d$.

  On the other hand, from the Extended Euclidean algorithm, we have
  $d=ra+sn$ for some $r,s\in\Z$.  So, 
  \[ d\equiv ra\pmod n \implies d\in \gen a \text{ in $\Zn$.}\]
\end{proof}  
\begin{prop}
  If $n,d\in\Z^+$ and $d\mid n$, then $|\gen d|=\frac n d$.
\end{prop}
\begin{proof}
  First note that for $k\in \{1,2\cots \frac n d -1\}$, $1\leq kd <
  n$, so these elements are non-zero in $\Zn$.  However 
 \[ \frac n d d = n \equiv 0\pmod n.\]
  So, $|d|=\frac n d$, which in turn implies $|\gen d|=\frac n d$.
\end{proof}

  A consequence of this proof is that when $d\mid n$, in $\Zn$ we have
  \[ \gen d = \left\{0, d, 2d\cots \left(\frac n d-1\right) \cdot d\right\}.\]

\begin{cor}\label{ordercor}
  If $n\in\Z^+$, $a\in \Z$ and $d=\gcd(n,a)$, then in $\Zn$,
  $|a|=\frac nd$.
\end{cor}
\begin{proof}
  Combining prior results, $|a|=|\gen a| = |\gen d| = \frac nd$.
\end{proof}

We can reframe some of these results as statements about congruences.
\begin{cor}
  If $n\in\Z^+$, $a,b\in\Z$, then the conguence
  \[  ax\equiv b\pmod n\]
  has a solution if and only if $\gcd(a,n)\mid b$.
\end{cor}
\begin{proof}
  The existence of a solution is equivalent to $b\in\gen a$ in $\Zn$,
  but $\gen a = \gen{\gcd(a,n)}$.
\end{proof}

We can connect this with the multiplication by $m$-map, $[m]$ on
$\Zn$.
\begin{prop}
  Suppose $m,n\in\Z^+$.  Let $d=\gcd(m,n)$.  Then the maps $[m]$ and
  $[d]$ have the same kernel, $\gen{n/d}$ which has order $d$, and the
  same image $\gen d$, which has order $n/d$.
\end{prop}
\begin{proof}
  The image of $[m]$ is $\{k[m]_n\mid k\in\Z\}=\gen m$.  By
  Proposition~\ref{imageszn}, $\Im([m])=\Im([d])$, and we get its
  order from Corollary~\ref{ordercor}.

  That $\ker([m])=\ker([d])$ follows from Corollary~\ref{znkernels}.
  
  Since $[d](n/d)=n\equiv 0\pmod n$, clearly $\frac n d \in
  \ker([d])$, and so $\gen{n/d}\subseteq \ker([d])$.  Conversely, if
  $a\in\ker([d])$, then $da\equiv 0\pmod n$.  So $n\mid da$, which
  implies $da=nk$ for some $k\in\Z$.  Thus $a=\frac n d k$, which
  implies $a\in\gen{n/d}$.  Thus, $\ker([d])=\gen{n/d}$.  The order of
  this subgroup is $n/(n/d)=d$ by Corollary~\ref{ordercor}.
\end{proof}

\subsection{Products}

As seen in the Section~\ref{vigenere}, if $G_1\cots G_n$ are groups,
then we can construct a new group with the set $G_1\times\cdots\times G_n$ with
operations done coordinatewise.  These groups are especially important
for us since elliptic curves give rise to finite abelian groups, which
are classified by the following theorem, which says that every finite
abelian group is isomorphic to a product of cyclic groups.
\begin{thm}[Classification of finite abelian groups]
\label{finiteabelian}
  If $A$ is a finite abelian group of order greater than $1$, then
  there exists a unique list of integers $n_1\cots n_k$ with $1<n_1$,
  $n_i\mid n_{i+1}$ for all $1\leq i<k$ such that 
  \[ A \cong \Z_{n_1}\times\cdots\times \Z_{n_k}.\]
\end{thm}
From the Theorem~\ref{crt} below, the Chinese Remainder Theorem, we
have isomorphisms
\[ \Z_{mn} \cong \Z_m\times \Z_n \quad\text{when $\gcd(m,n)=1$}.\]
This allows for an alternate statement of the theorem where each
cyclic group in the direct product has prime power order.  Both
versions are useful at different times.

We will defer the proof of Theorem~\ref{finiteabelian} to a group
theory course.  Nonetheless, it tells us that if we can count torsion
elements in a direct product of groups, then we can handle any finite
abelian group.

Suppose $(g_1\cots g_k)\in G_1\times\cdots \times G_k$.  Since
operations are done coordinatewise, if $j\in\Z^+$, then
\[ (g_1\cots g_k)^j = (g_1^j\cots g_k^j).\]
Thus,
\begin{align*}
(g_1\cots g_k)\text{ is $m$-torsion}
&\iff  (g_1\cots g_k)^m=(e_1\cots  e_k) \\
&\iff  (g_1^m\cots g_k^m)=(e_1\cots  e_k) \\
&\iff g_i^m=e_i \text{ for all $i$} \\
&\iff m \text{ is a multiple of $|g_i|$ for all $i$}
\end{align*}
This is the main part of the proof of the following.
\begin{prop}
If $G_1\cots G_n$ are groups and $(g_1\cots g_k)\in G_1\times\cdots
\times G_k$, then
\begin{enumerate}
\item $(g_1\cots g_k)$ has finite order if and only if $g_i$ has
  finite order for all $i$
\item if $(g_1\cots g_k)$ has finite order, then
\[ |(g_1\cots g_k)| = \operatorname*{lcm}_{1\leq i \leq k} |g_i|\]
\end{enumerate}
\end{prop}

If we have a product of abelian groups, $H\times K$ and $m\in\Z^+$,
then loosely the kernel of $[m]$ on $H\times K$ is the direct product
of the map on each factor, and similarly for the image.  To make a
formal statement, we will use $[m]_G$ to denote the multiplication by
$m$ map on $G$.

\begin{prop}\label{prodker}
  If $H$ and $K$ are abelian groups and $m\in\Z^+$, then
\[ \ker([m]_{H\times K}) = \ker([m]_H)\times \ker([m]_K)\]
and
\[ \Im([m]_{H\times K}) = \Im([m]_H)\times \Im([m]_K).\]
\end{prop}
By induction, this extends to a product of finitely many abelian
groups.
\begin{prob}
  Prove Proposition~\ref{prodker}.
\end{prob}

\section{Rings}
The definition of a ring applies to a couple of constructs we will
need, and acts as a stepping stone to fields.  However, there are only
a couple of examples which will be relevant.

\begin{defn}
  A set $R$ with two binary operations $+$ and $\cdot$ is a \emph{ring}
  if 
  \begin{enumerate}
  \item $R$ is an abelian group for $+$
  \item $\cdot $ is associative for $R$
  \item the left and right distributative laws hold for multiplication
    over addition, i.e., for all $a,b,c\in R$
    \[a\cdot(b+c) = a\cdot b+a\cdot c \text{ and } (b+c)\cdot a=b\cdot
    a+c\cdot a.\]
  \end{enumerate}
\end{defn}
\begin{remark}
  The order of operations in a ring are the same as in arithmetic:
  multiplication is done first, and then addition.
\end{remark}
All of our rings will satisfy the next property.
\begin{defn}
  A ring $R$ is a \term{ring with one} if $R$ has an identity for
  multiplication.
\end{defn}
Most of the our rings satisfy the next condition as well.
\begin{defn}
  A ring is a \term{commutative ring} if multiplication is
  commutative.
\end{defn}

\begin{exam}
  The sets $\Z$, $\Q$, $\R$, and $\C$ are all commutative rings with
  one with their usual addition and multiplication.
\end{exam}

\begin{exam}
  If $m\in\Z^+$, then $\Z_m$ is a commutative ring with $1$.
\end{exam}

\begin{remark}
  If a ring has one, then the identity element is unique by
  Proposition~\ref{unique-identity}.  We denote the identity for
  addition by $0_R$, or just $0$ for simplicity, and the identity for
  multiplication (for a ring with one) by $1_R$, or just $1$.
\end{remark}

\begin{exam}
  Let $n\in\Z^+$, then the set of $n\times n$ matrices $M_n(\R)$ is a
  ring with one (the identity matrix is the identity element for
  multiplication).  Note, it is \emph{not} a commutative ring if $n>1$.

  More generally, if $R$ is a ring, then $M_n(R)$ is also a ring.
  When adding and multiplying matrices, we just need to know how to
  add and multiply their entries, and one can check that the
  properties in the definition holds.  So, for $n\in\Z^+$, $M_n(\Z)$,
  $M_n(\Q)$,  $M_n(\C)$, and $M_n(\Z_m)$ are all rings.
\end{exam}
We prove one proposition about rings.
\begin{prop}\label{ring-zero}
  If $R$ is a ring, then for all $a\in R$,
  \[ 0\cdot a = a\cdot 0 = 0.\]
\end{prop}
We note that the proposition gives a property under multiplication for
the identity for addition.  The only part of the definition of the
axiom of ring which involves both operations is the distributive law,
so that plays a key role in the proof.  Of course, we also need to use
that $0$ is the identity for $+$, which comes via $0+0=0$.
\begin{proof}
  Let $a\in R$.  Then
  \[a\cdot 0 = a\cdot (0+0) = a\cdot 0+a\cdot 0.\]
  Adding $-(a\cdot 0)$ to both sides then gives
  \begin{align*}
-(a\cdot 0)+a\cdot 0=-(a\cdot 0)+a\cdot 0+a\cdot 0 &\implies
0 = 0+a\cdot 0\\
&\implies 0=a\cdot 0
\end{align*}
The proof that $0=0\cdot a$ is similar and left to the reader.
\end{proof}
\begin{prob}
  Complete the proof of Proposition~\ref{ring-zero} by proving
  $0=0\cdot a$.
\end{prob}

\begin{exam}
  Let $R=\{0\}$ as a subset of $\Z$.  We note that this one element
  set is a ring with the usual addition and multiplication.  It is
  called the \term{trivial ring}, and has the peculiar property that
  $0_R=1_R$.
\end{exam}
\begin{prob} \label{triv-ring-crit}
  Suppose $R$ is a ring with one where $0_R=1_R$.  Prove $R=\{0_R\}$,
  i.e., that $R$ is a copy of the trivial ring.
\end{prob}

\subsection{Units}
We start with a definition.
\begin{defn}
Let $R$ be a ring with one.  An element $a\in $ which has an inverse under
multiplication is called a \term{unit}.  
\end{defn}
By Proposition~\ref{unique-inverse}, the multiplicative inverse of a unit
is unique.  It is not hard to prove that the set of units forms a
group under multiplication.  In fact, the main thing to prove is that
the product of two units is a unit.
\begin{prob}
  Prove that if $R$ is a ring with one, then $R^\times$ is a group
  under multiplication.
\end{prob}
\begin{exam}
  Several groups under multiplication we have encountered before are
  special cases of this construction.  For example $\Qstar$, $\Rstar$,
  $\Cstar$, and $\Znstar$ are all examples.  
\end{exam}
Note, $R^\times$ need not be all non-zero elements, for example, as
seen in the case of $\Znstar$.
\begin{exam}
  In a matrix ring such as $M_n(\R)$, we let $M_n(\R)^\times =
  \GL_n(\R)$.  This is simply the invertible matrices.  More
  generally, if $R$ is a ring with one,
  \[ \GL_n(R)= M_n(R)^\times = \{A\in M_n(R) \mid \exists B\in M_n(R)
  \text{ s.t. } AB=BA=I_n\}.\]
  The point is that the inverse has to be in $M_n(R)$ as well.
\end{exam}

If $R$ is a commutative ring with one, then we can use determinants
since they only involve ring operations.  Then a consequence of
results from linear algebra is that
\[ \GL_n(R) = \{A\in M_n(R)\mid \det(A)\in R^\times\}.\]
\begin{exam}
  The two most relevant special cases here are
  \[ \GL_n(\Z) = \{A\in M_n(\Z)\mid \det(A)=\pm 1\}\]
  since $\Z^\times = \{1,-1\}$, and
  \[ \GL_n(\Z_m) = \{A\in M_n(\Z)\mid \det(A)\in \Z_m^\times\}.\]
  This last group is used in the Hill cipher.
\end{exam}


\subsection{Endomorphisms of groups}
Let $A$ be an abelian group written additively.
\begin{defn}
  An \term{endomorphism} of an abelian group $A$ is a homomorphism
  $f:A\to A$.  The set of all endomorphisms of $A$ is denoted by $\End(A)$.
\end{defn}
If $f,g\in\End(A)$, we can define their sum $(f+g):A\to A$ by function
addition.  I.e., define $(f+g)(a) = f(a)+g(a)$ for all $a\in A$.
\begin{prob}
  Prove that if $A$ is an abelian group and $f,g\in\End(A)$, then
  $f+g\in \End(A)$.
\end{prob}
Similarly, we can define a multiplication on $\End(A)$ by
composition.  Then the identity map acts as identity for
multiplication (i.e., compositition) and the trivial map which sends
every element to $0$ is the identity for addition.
\begin{prop}
  If $A$ is an abelian group, then $\End(A)$ is a ring with one.
\end{prop}

\section{Fields}

\begin{defn}
  A \term{field} $F$ is a commutative ring with one with $1_F\neq 0_F$,
  and every non-zero element is a unit.
\end{defn}
\begin{remark}
  As pointed out in problem~\ref{triv-ring-crit}, the condition
  $1_F\neq 0_F$ rules out the trivial ring from being considered a
  field.
\end{remark}

\begin{exam}
  With their usual operations, $\Q$, $\R$, and $\C$ are all fields,
  but $\Z$ is not because $2\neq 0$ in $\Z$, but $2$ has no
  multiplicative inverse in $\Z$.
\end{exam}

One of our most important examples of fields come from the rings $\Zn$.
\begin{prop} \label{zn-field}
  Let $n\in\Z^+$.  Then $\Zn$ is a field if and only if $n$ is prime.
\end{prop}
\begin{proof}
  If $n$ is prime and $a\in \Zn-\{0\}$, then $n\nmid a$.  Since $n$ is
  prime, $\gcd(a,n)=1$, so by the Bezout property (Thm.~\ref{bezout}),
  there exists $r,s\in\Z$ such that $ra+sn=1$, which implies $ra\equiv
  1\pmod n$.  Thus, every non-zero element is a unit in a commutative
  ring with one (and $n>1$ implies $0\not\equiv 1\pmod n$), and so
  $\Zn$ is a field.

  Conversely, if $\Zn$ is a field we must have $n>1$ (or else $0\equiv
  1\pmod n$).  If $n$ is not prime, then it is composite, and has a
  non-trivial factorization $n=ab$ with $1<a,b<n$.  Since $a\not\equiv
  0\pmod n$, it would have to have a multiplicative inverse $a'$ such
  that $aa'\equiv 1\pmod n$.  Multiplying by $b$ we get
  \begin{align*}
    b &\equiv b(aa') \pmod n \\
    &\equiv (ba)a' \pmod n \\
    &\equiv 0\cdot a' \pmod n \\
    &\equiv 0 \pmod n
  \end{align*}
  But then $1\equiv ab \equiv a\cdot 0 \equiv 0\pmod n$, a
  contradiction.  Thus, if $n$ is not prime, then $\Zn$ is not a field.
\end{proof}
We prove one small proposition for fields which forms the backbone of
many results involving roots of polynomials over a field.
\begin{prop}
  If $F$ is a field, $a,b\in F$ and $ab=0$, then $a=0$ or $b=0$.
\end{prop}
\begin{proof}
  Suppose $ab=0$ and $a\neq 0$.  Then $a$ has a multiplicative
  inverse; multiply by it to get
\begin{align*}
  a^{-1}(ab) = a^{-1}0 &\implies (a^{-1}a)b = 0 &\text{(by
    Prop.~\ref{ring-zero})} \\
  &\implies 1\cdot b = 0 \\
  &\implies b=0.
\end{align*}
\end{proof}
\begin{remark}
  The conclusion fails in some of the rings we have seen.  For
  example, in $\Z_6$, $2\cdot 3 \equiv 0\pmod 6$, but $2\not\equiv
  0\pmod 6$ and $3\not\equiv 0\pmod 6$.  Similarly, in $M_2(\R)$,\
  \[ \begin{pmatrix} 0 & 1\\0 &0\end{pmatrix}\begin{pmatrix} 0 & 1\\0
    & 0\end{pmatrix} = \begin{pmatrix} 0 & 0\\0 &0\end{pmatrix}\]
  but neither matrix on the left-hand side is the zero-matrix.
\end{remark}



\subsection{Characteristic}
The notion of characteristic can be applied to any ring, but will work
in the context of rings with one.
\begin{defn}
  Let $R$ be a ring with one.  If $1_R$ has infinite order in the
  additive group, we say $R$ has \term{characteristic zero}.  Otherwise, the
  \term{characteristic} of $R$ is the order of $1_R$ under addition.  We
  denote the characteristic of $R$ by $\chr(R)$.
\end{defn}
Recall that in a additive group, we defined $n\cdot a$ where
$n\in\Z^+$ to be $a\pots a$ ($n$ times).
\begin{prob}
  Suppose $R$ is a ring with one of characteristic $n>0$.  Prove that
  for all $a\in R$,
  $n\cdot a=0_R$, and that $n$ is the smallest positive integer with
  this property.
\end{prob}
\begin{exam}
  Some of the most familiar rings have characteristic $0$:
\[ 0= \chr(\Z)=\chr(\Q)=\chr(\R)=\chr(\C). \]
\end{exam}
\begin{exam}\label{zn-char}
  If $n\in\Z^+$, then $\chr(\Zn)=n$.
\end{exam}
Suppose $R$ and $S$ are rings with one, $R$ a subring of $S$ with
$1_R=1_S$.  Since the characteristic only depends the multiplicative
identity, we immediately get
\begin{prop}
Suppose $R$ and $S$ are rings with one, $R$ a subring of $S$ with
$1_R=1_S$, then $\chr(R)=\chr(S)$.
\end{prop}
We will be mainly interested in fields, and the characteristic of a
field is restricted.
\begin{prop} \label{field-char}
  If $F$ is a field, then either $\chr(F)=0$ or $\chr(F)=p$ for some
  prime $p$.
\end{prop}
\begin{proof}
  Suppose $F$ has finite characteristic $n$.  If $n=1$, then $1\cdot
  1_F=0_F$, which implies $1_F=0_F$ \con.  Thus, $n>1$.

  If $n$ is not prime, then it is composite and $n=ab$ with
  $1<a,b<n$.  From the distributive law
%  \[ \begin{array}{rcl@{}l}
%    1\pots 1 & = & (1\pots 1)&(1\pots 1) \\
\[ 0=\underbrace{1\pots 1}_\text{$n$ times}  =  (\underbrace{1\pots
  1}_\text{$a$ times}) (\underbrace{1\pots 1}_\text{$b$ times})\]
But in a field, then implies $a\cdot 1=0$ or $b\cdot 1=0$.  Either
way, we get a contradiction to the fact that $n$ is the additive order
of $1$ \con.
\end{proof}
\begin{remark}
  Proposition~\ref{field-char} is consistent with the
  example above and Proposition~\ref{zn-field} since the
  latter two tell us $\chr(\Zn)=n$, and $\Zn$ is a field iff $n$ is
  prime, so $\Zn$ is a field iff $\chr(\Zn)$ is prime.
\end{remark}

\section{Chinese remainder theorem}
The Chinese remainder theorem is probably familiar from elementary
number theory or cryptography.  Here we phrase it in a way to
highlight it connections to abstract algebra.
\begin{thm}[Chinese remainder theorem] \label{crt}
  Let $m,n\in\Z^+$ such that $\gcd(m,n)=1$.  Then the map
  \begin{align*}
    \phi:\Z_{mn}&\to \Z_m\times\Z_n \\
    [a]_{mn} &\mapsto ([a]_m, [a]_n)
  \end{align*}
  is a ring isomorphism.
\end{thm}
Recall that being an isomorphism means that we can go back and forth
between the two sides.  Since it is an isomorphism of rings, the map
respects addition and multiplication.  A consequence of the latter is
the multiplicative property of the Euler phi-function.
\begin{cor}
  If $m,n\in\Z^+$ with $\gcd(m,n)=1$, then
  \[ \varphi(mn) = \varphi(m)\varphi(n).\]
\end{cor}
\begin{proof}[Outline of proof of corollary:]
Filling in the steps is left to the reader.  Recall that if $R$ is a
ring with one, then $R^\times$ denotes its set of units.
\begin{enumerate}
  \item If $\phi:R\to S$ is a ring isomorphism between rings with one,
    then it gives a bijection between $R^\times$ and $S^\times$.
  \item If $R$ and $S$ are commutative rings with $1$, then $(R\times
    S)^\times =R^\times\times S^\times$.
  \item Deduce $|\Z_{mn}^\times| = |\Z_m^\times|\cdot |\Z_n^\times|$.
  \item Use that $|\Z_n^\times| = \varphi(n)$.
\end{enumerate}
\end{proof}

\begin{proof}[Proof of CRT]
  We sketch the proof.
  \begin{enumerate}
  \item Show that the map $\phi$ is well-defined by proving that if $a,b\in\Z$,
    $a\equiv b\pmod{mn}$, then $a\equiv b\pmod m$ and $a\equiv b\pmod
    n$.
  \item Show that $\phi$ respects addition (this is mainly invoking a 
    the definition of addition in $\Zn$).
  \item Show that $\phi$ respects multiplication (this is mainly invoking a 
    definition of multiplication in $\Zn$).
  \item Show that the map is one-to-one by showing $\ker(\phi) =
    \{[0]_{mn}\}$.
  \item Deduce that $\phi$ is surjective by counting (and using that
    it is injective).
  \end{enumerate}
\end{proof}
\begin{prob}
  \begin{enumerate}
  \item[(a)]  Fill in the details of the proof of Theorem~\ref{crt}.  
  \item[(b)] Which  step(s) use the hypothesis that $\gcd(m,n)=1$?
  \end{enumerate}
\end{prob}
The traditional version of the Chinese Remainder Theorem states that
if $a,b\in\Z$, $m,n\in\Z^+$ with $\gcd(m,n)=1$, then the system of
congruences 
\begin{align*}
x&\equiv a \pmod m \\
x &\equiv b\pmod n
\end{align*} 
has a unique solution modulo $mn$.  The existence of this solution is
equivalent to the statement that the map $\phi$ in Theorem~\ref{crt}
is surjective, and the uniqueness is then equivalent to the statement
that $\phi$ is injective.  However, the the traditional version
does not draw out the algebraic nature of the correspondence -- that
$\phi$ is a ring homomorphism -- which is to many applications.






\section{Curves}
We work in the realm of algebraic geometry.  This means that some of
our basic objects come as the set of solutions to systems of
polynomial equations.

In linear algebra, one studies these objects in the case when all of
the equations are linear.  If we have $n$ linear equations in $m$
unknowns with $n\leq m$, then in general we get a solution set with
$m-n$ free parameters.  It is an $m-n$ dimensional set (a translate of
an $m-n$ dimensional vector space).  However, that is not the only
possible outcome: one might end up with more free parameters, or no
solutions at all.

We will work with non-linear equations, but make an important
simplifying assumption: we virtually always work with one equation in
two unknowns.  In the linear case, this resolves the ambiguity and the
solution set is a line.  In the non-linear case, we generally end up
with something one-dimensional.  There are two degenerate
situations we largely avoid because they are
distractions from our main interest.
\begin{exam}
  The solution set for $x^2+y^2=-1$ over $\R$ is empty.
\end{exam}
\begin{exam}
  The solution set for $x^2=y^2$ over $\R$ is the union of two lines.
  If we move everything to one side of the equation, it factors:
  $(x+y)(x-y)=0$.  In a case like this, we say the solution set is
  reducible.  If the corresponding polynomial does not factor, we say
  that the solution set is irreducible.
\end{exam}
Again, we are not interested in these degenerate cases.  In practice, the
solution set to the equations we work with will be one dimensional and
irreducible.  The result is then called a \emph{curve}.



\section{Projective space}

\subsection{Geometry}
When studying functions in calculus and high school algebra, we
consider their graphs in the plane.  Here the two axes play different
roles, and sometimes particular points in the plane are special.  For
example, one might consider the graph of a polynomial, and places
where it hits the $x$-axis are special because they correspond to
roots.

In these courses, analytic geometry refers to using a coordinatized
plane to study geometric ideas.  A geometric figure is just a subset
of the plane, and an essential notion of when two geometric figures
are the same is if they are congruent.  In particular, sets are
congruent to their images under rotation, reflection, or translation
(shift); there is no special direction, and the origin is just another
point.

We will take the geometric point of view of the plane.  As in
calculus, we may consider higher dimensional space as well.  We will
work over a field $K$, and define \emph{affine $n$-space} to be $K^n$
as a generalization of $\R^n$.

\subsection{Slopes in the plane}
Consider lines in $\R^2$.  An essential invariant is slope.
From the geometric point of view, vertical lines are as good as any
other line, so the set of possible slopes is almost given by $\R$, but
we need one more ``slope'' to account for vertical lines.  Nothing
changes algebraically if we generalize to $K^2$ where $K$ is a field,
so we do just that.

We can approach this another way.  Every slope is represented by a
single line through the origin, the set of lines through the
origin could be used as the set of slopes.  Each such line is
completely determined by a point it goes through $(a,b)\neq (0,0)$.
We define a relation on these points based on when they determine the
same line:
\begin{equation} \label{line-eq-rel}
 (a,b)\sim (c,d) \iff (a,b)=\lambda (c,d)\quad\text{for some
  $\lambda\in K$.}
\end{equation}
Here $\lambda(c,d)= (\lambda c, \lambda d)$ as in scalar
multiplication from linear algebra.
\begin{prob}
  Prove $\sim$ in equation~\eqref{line-eq-rel} defines an equivalence
  relation.
\end{prob}
Since this is an equivalence relation, we want to consider the
equivalence classes and write $[a:b]$ for the
equivalence class of $(a,b)$.  We define $\pr^1(K)$ to be the set of
equivalence classes.  Then the lines through the origin are
in one-to-one correspondence with these equivalence classes.

Naturally, we can connect this point of view with the original one,
that the set of slopes should be $K$ plus one more element for
vertical lines.
\begin{prop}\label{p1}
  If $K$ is a field, the elements of $\pr^1(K)$ are in one-to-one
  correspondence with
  \[ \{[1:m]\mid m\in K\}\cup \{[0:1]\}.\]
\end{prop}
The proof is straightforward if one breaks it into cases for $[a:b]$
based on whether or not $a=0$.
\begin{prob}
  Prove Proposition~\ref{p1}.
\end{prob}

\subsection{$\pr^n(K)$}
We now generalize the construction from the previous section.  
Points in $\pr^n(K)$ to correspond to lines through the origin in
$K^{n+1}$.  Let
\[ S= K^{n+1}-\{(0\cots 0)\}.\]
As above, we define a relation 
\[ \vec u\sim \vec v \iff \vec u=\lambda\vec v \text{ for some $\lambda\in 
K^\times$}\]
where if $\vec u=(x_0\cots x_n)$, then $\lambda\vec u = (\lambda x_0\cots
\lambda x_n)$.
We then let $\pr^n(K)=S/\sim$.

As above, we let $[x_0\ccots x_n]$ denote the equivalence class of
$(x_0\cots x_n)$.

\subsection{Affine points and points at infinity}
We can partition the points of $\pr^n(K)$ into two sets in a
natural way.  Fix $i\in\{0\cots n\}$.  The two sets are
\[ \{[x_0\ccots x_n] \mid x_i\neq 0\} \text{ and } 
\{[x_0\ccots x_n] \mid x_i = 0\}. \]
Although the coordinates of $[x_0\ccots x_n]$ are only well-defined
up to a non-zero scalar, the condition $x_i=0$ makes sense.

If $[x_0\ccots x_n]$ is in the first set, we can normalize the
scaling by multiplying through by $\frac1{x_i}$.  This makes the
$i$th coordinate $1$, and there is only one way to represent such
a point with $x_i=1$.  There are $n$ coordinates $x_j$ with $j\neq i$,
so there is a natural bijection between the first set and $K^n$, i.e.,
a copy of affine $n$-space.

Points in the second set have $x_i=0$.  There are $n$ remaining coordinates,
and they
\begin{enumerate}
\item cannot all be zero
\item give the same point in projective space if they differ by
multiplication by a non-zero element of $K$
\end{enumerate}
But, these two conditions describe projective $n-1$ dimensional space.
In other words, dropping $x_i$ (which equals $0$) from the list gives
a natural bijection between the second set and $\pr^{n-1}(K)$.

So, we can break $\pr^n(K)$ into a disjoint union:
\[ \pr^n(K) = K^n \cup \pr^{n-1}(K). \]
The $K^n$ are called the affine points, and the $\pr^{n-1}(K)$ are
the points at infinity.  Note, we initially picked $i\in\{0\cots n\}$,
so there are $n+1$ choices for $i$, and each gives a different decomposition.
A given point $[x_0\ccots x_n]$ ends up being an affine point in the 
decomposition for $i$ if $x_i\neq 0$, and is a point at infinity where
$x_i=0$.

On one hand, a point like $[1:2:3]\in\pr^2(\R)$ is an affine point in all
three of the decompositions, the point $[1:0:2]$ is an affine point in two
decompositions and gives a point at infinity for the third (when $i=1$).
Importantly, every point $[x_0\ccots x_n]\in\pr^n(K)$ has some $x_i\neq 0$
(from the original construction), so every point is an affine point for
at least one of the decompositions.

\subsection{Homogenization and dehomogenization}
Here we look at the process of switching between affine and projective
coordinates.  An elliptic curve lives in projective space, so we need
to use projective coordinates sometimes. However, any individual point
lives in at least one affine subset of projective space, so we can
switch to affine coordinates when dealing with that point.

We illustrate the process with $\pr^2(\R)$.  For projective
coordinates, we will use capital letters: $[X:Y:Z]$.  There are three
affine spaces covering $\pr^2(\R)$, but our favorite will be the set
where $Z\neq 0$.  On this set, the transition between affine and
projective coordinates is
\[ (x,y) \leftrightarrow \left[\frac X Z : \frac Y Z : 1\right].\]
In other words, $x=X/Z$ and $y=Y/Z$.

\subsubsection{Homogenization}

Suppose we start with the affine equation 
\begin{equation}\label{ex1}
x^2+3y^2 = 1,
\end{equation}
and want to convert it to projective coordinates.  We simply make the
substitutions above, and clear denominators.  This is what it looks
like.
\begin{equation} \label{eq1hom}
 \left(\frac X Z\right)^2 + 3\left( \frac Y Z\right)^2 = 1 \iff
X^2+3Y^2=Z^2.
\end{equation}
The end result is a \emph{homogeneous equation} in that every term has
total degree $2$ (the \emph{total degree} of a monomial is the sum of
the powers of all of its variables).  Because of this the process of
going from affine to projective coordinates is called homogenization.

Note that if $f(X,Y,Z)=0$ is homogeneous of degree $d$, then every
monomial has total degree $d$, so $f(\lambda X, \lambda Y, \lambda Z)=
\lambda^d f(X,Y,Z)$.  So, if one representative of $[a:b:c]$ is a
solution to $f(X,Y,Z)=0$, then so is every representative $(\lambda a,
\lambda b, \lambda c)$.  So it is fitting to use homogeneous equations
when using projective coordinates.

\begin{exam}
  Here we find the projective equation for $x^3+xy+2x -3=y^2$:
  \[ \left(\frac X Z\right)^3 + \left(\frac X Z\right) \left(\frac Y
    Z\right) +2 \left(\frac X Z\right) - 3 = \left(\frac Y Z\right)^2
\]
 which, after clearing denominators gives
\[X^3+XYZ + 2XZ^2 -3Z^3 = Y^2Z\]
\end{exam}

There is a slightly faster way to homogenize.  Suppose an equation has
total degree $d$ (i.e., the maximum of the degrees of monomials is
$d$).  For a term with degree $k$, the substitution introduces $Z^k$
to its denominator.  To clear all denominators, we multiply by $Z^d$,
so the degree $k$ term is multiplied by $Z^{d-K}$.  So, the term is
multiplied by the power of $Z$ to bring it to the maximum degree.

The process is essentially the same with more variables, except that
we may go back to $X_0\cots X_n$.
\begin{exam}
  To find the projective equation for
  $x_1^7-3x_2x_3^2=4x_1^3x_4^5-11$, we use the faster method:
  capitalize all of the variables and mulitply each term by a power of
  $X_0$ so that its degree matches the largest degree, which is $8$
  (from the $4x_1^3x_4^5$ term):
  \[ X_0X_1^7-3X_0^5X_2X_3^2=4X_1^3X_4^5-11X_0^8.\]
\end{exam}
\subsubsection{Dehomogenization}
Dehomogenization is simple.  If we want to dehomogenize with respect
to the variable $x_i$, we use the correspondence
\[ (x_0\cots x_{i-1}, x_{i+1}\cots x_n) \leftrightarrow [X_0\ccots
X_{i-1}:1:X_{i+1}\ccots X_n].\]
So, we replace $X_i$ with $1$, and convert the other variables
accordingly.
\begin{exam}
  To dehomogenize $ X_0X_1^7-3X_0^5X_2X_3^2=4X_1^3X_4^5-11X_0^8$ with
  respect to $X_2$, we apply the process to get
  \[ x_0x_1^7-3x_0^5x_3^2=4x_1^3x_4^5-11x_0^8.\]
\end{exam}

There is a danger when homogenizing and dehomogenizing equations.  In
the examples above, we started with $x_1^7-3x_2x_3^2=4x_1^3x_4^5-11$,
homogenized and dehomogenized to get
$x_0x_1^7-3x_0^5x_3^2=4x_1^3x_4^5-11x_0^8$.  However, variables in
these two equations represent different things.  In the first
equation, $x_1=X_1/X_0$ and in the second equation $x_1=X_1/X_2$.  The
projective coordinates are consistent, but not the two sets of affine
coordinates.  For this reason, we will usually use different variable
names for different dehomogenizations.

Usually, we work in $\pr^2$ with coordinates $[X:Y:Z]$.  The affine
coordinates with $Z=1$ will be $(x,y)$, which means we need to pick
other letters for the other dehomogenizations.
\begin{exam}
  Dehomogenize $X^3+XYZ + 2XZ^2 -3Z^3 = Y^2Z$ with respect to $Y$.
  Here we will use $u=X/Y$ and $v=Z/Y$.  The process is to set $Y=1$,
  replace $X$ with $u$, and replace $Z$ with $v$, which gives
  $u^3+uv+2uv^2-3v^3 = v$.
\end{exam}

\subsubsection{Points at infinity}

Here, we will work in $\pr^2$.  As a first example, consider
the line $y=mx+b$.  It has homogenization is $Y=mX+bZ$.

Points at infinity are points which satisfy the homogenized equation
in projective space which do not come from solutions to the
original equation, i.e., they have $Z=0$.  So, we set $Z=0$ which
gives $Y=mX$.  Note, if one point is a solution to
this equation, then all of its scalar multiples are solutions, but
these multiples are all representatives of the same projective point.
So, we are only interested in solutions up to scalar multiplication.
In this case, we get one point: $[1:m:0]$.

Note, any two lines with slope $m$ hit the same point at infinity.  If
we consider vertical lines, then a similar computation shows that they
have one point at infinity, namely $[0:1:0]$.
\begin{prob}
  Show that the line $x=c$ has one point at infinity, namely
  $[0:1:0]$.
\end{prob}

This points out part of the appeal of working in projective space.
\begin{itemize}
\item  In the affine plane, two distinct lines intersect in $0$ or $1$
  point.
\item  In the projective plane, two distinct lines always intersect in $1$
  point.
\end{itemize}
Moreover, there we see that the points at infinity are in $1-1$
correspondence with slopes of lines (including the infinite slope).

Next we consider $x^2-y^2=1$, a hyperbola with asymptotes $y=\pm x$.
Intuitively, we expect the points at infinity to match those of its
asymptotes, and that is exactly what happens.  

The homogenization of the equation is $X^2-Y^2=Z^2$.
Setting $Z=0$ gives 
\[ X^2-Y^2=0 \iff Y=\pm X. \]
Up to scalar multiplication, this gives get two projective points:
$[1:1:0]$, and $[1:-1:0]$.  These match the points at infinity for the
two asymptotes.

As a final example, consider the equation
$x^2+3y^2=1$.  We saw earlier that its homogenization is
$X^2+3Y^2=Z^2$.  Setting $Z=0$ and solving gives
\[  X^2+3Y^2=0 \iff X^2=-3Y^2. \]

To deal with the fact that this is a homogeneous equation with
projective solutions, we can dehomogenize.  We just need to be careful
to not miss any solutions in the process.

We already have $Z=0$ so we do not get a projective point from the
obvious solution $X=Y=0$.  Note that if $Y=0$, then $X=0$ \con, so
$Y\neq 0$.  We can then dehomogenize with respect to $Y$ to get the
equation $u^2=-3$ (with $u=X/Y$).

Over $\R$, we get no solutions, so there are no points at infinity in
$\pr^2(\R)$.  However, over $\C$ there are solutions, $u=\pm i\sqrt
3$.  Going back to projective coordinates, this gives two points
$[\sqrt{3}i:1:0]$ and $[-\sqrt{3}i:1:0]$. 

\subsubsection{Quadratic curves}

By a quadratic curve, we mean (in affine coordinates) the solutions to
a quadratic equation in two variables.  In analytic geometry, one
learns how to make coordinate changes to shift a central point to the
origin, and then to rotate the graph until it takes one of three
forms: parabola, hyperbola, or ellipse\footnote{We are ignoring the
  degenerate cases here, like $x^2=y^2$.}.  If we allow scaling of
variables, then every quadratic equation can be transformed into one
of three specific equations.

We considered the projective equation $X^2+Y^2=Z^2$.
\begin{itemize}
\item If we dehomogenize with respect to $Z$, we get $x^2+y^2=1$, an ellipse
\item If we dehomogenize with respect to $Y$, we get $u^2+1=v^2\iff u^2-v^2=1$, a hyperbola
\end{itemize}
The affine curves are parts of a single curve which may be missing
points.  Here, the ellipse and the hyperbola are just two views of the
same curve.  For points over $\R$, the ellipse has no points at
infinity, but the hyperbola has two points at infinity.  We think of
those points as connecting the branches of the hyperbola.

The next problem shows that a curve can have one affine part be a
parabola, and another be a hyperbola.
\begin{prob}
  Start with $y=x^2$.
  \begin{enumerate}
  \item Find its homogeneous equation.
  \item Show that it has one point at infinity (over any field).
  \item Dehomogenize with respect to $X$.
  \end{enumerate}
\end{prob}

From the examples above, we have seen that a parabola and a hyperbola
can both be affine parts of the same curve.  Similarly, an ellipse and
a hyperbola can be affine parts of the same curve.  There are three
affine parts to every projective curve, and in fact, one can find
equations of curves where these three affine parts hit all three
types!  In projective space, the situation is very simple: there is
only one type of (smooth) quadratic equation.  From the projective
curves point of view, when you look at any affine piece, you see most
of the curve, but not all of it.

The situation is analgous to holding a spherical object and trying to
see it.  No matter how you hold it, you do not see all of the surface
of the sphere.  You can, however, turn the sphere and get different
views.  Every point is visible from some view, but no one view lets
you see all of the points at once.

Similarly with a curve in projective space (of any degree), the affine
parts let us see parts (most) of the curve, and every point is in part
of some affine piece, but no affine part contains all of the points
(technically, we should add the condition ``over an algebraically
closed field'').

\section{More about group homomorphisms}
%\subsection{Counting inverse images}
\label{coset-home}

For general functions between sets, the inverse image of an element
can have various sizes.  For group homomorphisms, it is much more
restrictive.  Suppose $f:H\to K$ is a group homomorphism.  If $k\in K$
but $k\not\in\Im(f)$, then clearly $f^{-1}(k)=\emptyset$.  If $k\in
\Im(f)$, then $k=f(a)$ for some $a\in H$.  We focus on these elements.

\begin{prop} \label{invimgcount}
  Suppose $f:G_1\to G_2$ is a homomorphism of groups.  Then for all $k\in
  \Im(f)$, the sets $f^{-1}(k)$ have the same number of elements as
  $\ker(f)$.
\end{prop}
We will prove this in two steps: we show that the inverse image of a
point (from the image) is always a left coset, and then we show that
all left cosets (of a given subgroup) have the same number of
elements.  First, we should define left coset.
\begin{defn}
  If $G$ is a group, $H$ is a subgroup of $G$, and $a\in G$, then the
  \term{left coset} of $a$, denoted $aH$ is the set
  \[ aH=\{ah\mid h\in H\}.\]
\end{defn}
Recall that if $f:G_1\to G_2$ is a homomorphism, then $\ker(f)$, the
kernel of $f$, is a subgroup of $G_1$.  Moreover, $\ker(f) = f^{-1}(e)$,
so it is the inverse image of a particular point.
\begin{lemma}
  Suppose $f:G_1\to G_2$ is a homomorphism of groups and $a\in G_1$.  Then
  \[ a\ker(f) = f^{-1}(f(a)).\]
\end{lemma}
\begin{proof}
  Suppose $x\in a\ker(f)$.  Then $x=ah$ for some $h\in\ker(f)$.  Note
  $f(h) =e$.  So, 
\[ f(x) = f(ah) = f(a)f(h) 
     \implies f(x) = f(a)\cdot e 
     \implies f(x)= f(a). \]
This implies $x\in f^{-1}(f(a))$.  

Now suppose $x\in f^{-1}(f(a))$.  Then $f(x)=f(a)$, which implies
\[ e=f(x)f(a)^{-1} = f(x) f(a^{-1}) = f(xa^{-1}).\]
Thus, 
$$xa^{-1}\in \ker(f) \implies xa^{-1} = h \text{ for some $h\in\ker(f)$}.$$
Therefore, $x=ah\in a\ker(f)$.
\end{proof}
\begin{lemma}
  If $H$ is a subgroup of a group $G$, then for all $g\in G$, $|gH| =
  |H|$.
\end{lemma}
  We show that these sets have the same number of elements by 
  constructing a bijection between them.\footnote{This works in the situation
  we are interested in, namely when $H$ is finite, but applies to
  cardinalities of infinite sets.}
\begin{proof}
  Let $g\in G$.  Define
\begin{align*}
  f:H&\to gH \\
  h&\mapsto gh
\end{align*}
It is clear from the definition of $gH$ that this function is surjective.
To see that it is injective, note that for all $a,b\in H$,
\[  f(a) = f(b) \implies ga = gb \implies a=b\]
by left-cancellation (or multiplication on the left by $g^{-1}$.  Thus
$f$ is also injective, hence bijective, and we get $|H|=|gH|$.
\end{proof}


\section{More about fields}
\subsection{Extension fields}

We stipulate the following as a fact from abstract algebra.
\begin{prop}
  If $K$ is a field, then there is a field $\Kbar$ called the {\em
    algebraic closure of $K$} with the following two properties:
  \begin{enumerate}
  \item every $\alpha\in \Kbar$ is the root of a non-constant polynomial with
    coefficients in $K$
  \item every non-constant polynomial with coefficients in $K$ factors
    as a product of degree $1$ polynomials over $\Kbar$.
  \end{enumerate}
\end{prop}
The theorem covers not only the existence of this large field
containing all roots of all polynomials from $K$, but also that every
element of this field is the root of a polynomial.  In particular, if
$\alpha$ and $\beta$ are both roots of polynomials in $K[x]$, then so
is $\alpha+\beta$ and $\alpha\beta$.


\begin{exam}
  If $K=\R$, then $\Kbar=\C$.
\end{exam}

When considering roots of polynomials, we can multiply by a non-zero
constant.  So, we normally deal with \term{monic} polynomials, i.e.,
ones whose highest degree coefficient is $1$.

Similarly, when considering the factorization of monic polynomials, we
can adjust constant factors to make each factor monic as well.
\begin{exam}
  The polynomial $x^2-4$ factors over $\Q$ as $(2x-4)(\frac12 x+1)$,
  but we can multiply the first factor by $\frac12$ and the second by
  $2$ to obtain $x^2-4=(x-2)(x+2)$.
\end{exam}
\begin{prop}
  Let $K$ be a field and $\alpha\in\Kbar$.  Let $f(x)$ be a
  monic polynomial with coefficients in $K$ of minimal degree
  such that $f(\alpha)=0$.  Then
  \begin{enumerate}
  \item $f(x)$ is irreducible over $K$
  \item if $\deg(f) = n$, then the elements $a_{n-1}\alpha^{n-1}\pots
    a_0$ with $a_i\in K$ are distinct.
  \end{enumerate}
\end{prop}
\begin{proof}
  For the first part, suppose not.  Then we have a non-trivial
  factorization $f(x)=g(x)h(x)$, and can assume $g(x)$ and $h(x)$ are
  both monic with coefficients in $K$.  Since the factorization is
  non-trivial, both $g(x)$ and $h(x)$ have degree smaller than the
  degree of $f(x)$.  Evaluating at $\alpha$ we get
  \begin{equation} \label{eq-prod-0}
    0 = f(\alpha) = g(\alpha)h(\alpha)
  \end{equation}
  and so $g(\alpha)=0$ or $h(\alpha)=0$ (since
  equation~\eqref{eq-prod-0} takes place in the field $\Kbar$).  This
  contradicts the minimality of the degree of $f(x)$ \con.

  For the second part, suppose
  \[ a_{n-1}\alpha^{n-1}\pots a_0 = a_{n-1}'\alpha^{n-1}\pots a_0'\]
  for some $a_i, a_i'\in K$.  Subtracting, we get
  \[(a_{n-1}-a_{n-1}')\alpha^{n-1}\pots (a_0-a_0')=0.\]
  By the minimality of the degree of $f(x)$, $\alpha$ cannot be the
  root of a polynomial of degree less than $n$ over $K$, so
  $(a_{n-1}-a_{n-1}')x^{n-1}\pots (a_0-a_0')$ must be the
  $0$-polynomial, i.e., all of its coefficients are zero.  Thus,
  $a_i=a_i'$ for all $i$.
\end{proof}
It is not hard to show that the polynomial $f(x)$ in the previous
proposition is unique for $\alpha$ (lowest degree monic polynomial
over $K$ with $\alpha$ as a root), and is called the \emph{monic
  irreducible polynomial of $\alpha$}.

Another result from algebra:
\begin{prop} \label{ext-field}
  If $K$ is a field, $\gamma\in \Kbar$, and $n$ the degree of the monic
  irreducible polynomial for $\gamma$.  Then
  \[ K[\gamma]= \{a_{n-1}\gamma^{n-1}\pots a_0 \mid a_i\in K\}\]
  is a subring of $\Kbar$ which is a field.
\end{prop}
\begin{proof}
  Let $\alpha=\sum_{i=0}^{n-1} a_i \gamma^i, \beta=\sum_{i=0}^{n-1} b_i \gamma^i\in
  K[\gamma]$.  Then
  \[ \alpha+\beta=\sum_{i=0}^{n-1} a_i \gamma^i+\sum_{i=0}^{n-1} b_i
  \gamma^i =\sum_{i=0}^{n-1} (a_i+b_i) \gamma^i \in K[\gamma].\]
Similarly,
\[ -\left(\sum_{i=0}^{n-1} a_i \gamma^i\right)=\sum_{i=0}^{n-1} (-a_i)
\gamma^i\in K[\gamma],\]
and clearly $0=\sum_{i=0}^{n-1} 0\cdot \gamma^i\in K[\gamma]$.  To see
that $K[\gamma]$ is a subring of $\Kbar$, all that is left is to show
that it is closed under multiplication.

From the distributative law,
\begin{equation} \label{alphabeta}
 \alpha\cdot\beta = \sum_{i=0}^{2n-2} c_i \gamma^i
\end{equation}
for some $c_i\in K$.  We can reduce this to an element of the right
form as follows.  Note $f(x)=x^n+d_{n-1}x^{n-1}\pots d_0$ for some
$d_i\in K$, so $f(\gamma)=0$ implies
\[ \gamma^n = -d_0-d_1\gamma-\cdots -d_{n-1}\gamma^{n-1},\]
and multiplying by $\gamma^k$:
\begin{equation}\label{gammapow}
 \gamma^{n+k} = -d_0\gamma^k-d_1\gamma^{k+1}-\cdots
-d_{n-1}\gamma^{n+k-1}.
\end{equation}
In a sum such as the right-hand side of equation~\eqref{alphabeta}, we
can reduce the ``degree in $\gamma$'' by repeatedly making a
substitution from equation~\eqref{gammapow} until the degree is less
than $n$.

To complete the proof, we would need to show that every non-zero
element of $K[\gamma]$ has a multiplicative inverse of the same form.
We leave the proof to a course in abstract algebra.\footnote{One could
  either prove and use an analogue of the Bezout property for
  polynomials, or linear algebra.}
\end{proof}
Now suppose $K$ is a field, $\alpha\in\Kbar$, and $f(x)$ is the monic
irreducible polynomial for $\alpha$ over $K$.  By
Prop.~\ref{ext-field}, we get a field $K[\alpha]$.  It is pretty easy
to see that elements of that form are closed under addition and
additive inverse.  When multiplying two elements of that form, we
typically get powers of $\alpha$ as high as $\alpha^{2n-2}$.

%\subsection{Fields and polynomials}
% Let $F$ be a field and $\alpha\in\overline F$.  A polynomial over $F$
% is a polynomial $f=a_nx^n\pots a_0$ where all $a_i\in F$.  Then $f$ is
% non-zero if some $a_i\neq 0$.

% From the definition of $\overline F$, $\alpha$ is a root of a non-zero
% polynomial $g$.  We can then repeatedly factor $g$ as a product of
% polynomials of lower degree until we get to irreducibles: $g=h_1\cdots
% h_m$.  Then $0=g(\alpha)=h_1(\alpha)\cdots h_m(\alpha)$, so
% $h_i(\alpha)=0$ for some $i$.  Thus $\alpha$ is a root of an
% irreducible polynomial.

% Now let $g$ be an irreducible polynomial with $\alpha$ as a root.  We
% can divide $g$ by its leading coefficient, and reduce to the case
% where the leading coefficient of one.  A polynomial with leading
% coefficient of one is called \term{monic}.

\subsection{Finite fields}

A \emph{finite field} is simply a field $K$ which is a finite set.
The most familiar examples are $\Z_p$ with $p$ prime.  In general,
one can prove
\begin{itemize}
\item if $K$ is a finite field, then it has characteristic $p$
 for some prime $p$
\item if $K$ is a finite field of characteristic $p$, then it
has order $p^n$ for some $n\in\Z^+$
\item for every prime $p$ and positive integer $n$, there exists a
  unique subfield of $\Fpbar$ which is finite of order $p^n$.
\end{itemize}
Given this, we will write $\F_{p^n}$ for the field of order $p^n$.
Since $\Z_p$ is a field of order $p$, we have $\Z_p=\F_p$.  We will
use the $\F_p$ notation when we want to emphasize that we are talking
about $\Z_p$ as a field, and $\Z_p$ if we want to emphasize that we
are dealing with integers modulo $p$, or that we have a cyclic group
of order $p$.

\subsection{Characteristic $p$}
Let $F$ be a field with prime characteristic $p$.  In particular, this
applies to any finite field.  Our goal here is to prove the following
proposition, and see some of its consequences.
\begin{prop}[Freshman Dream]
\label{dream}
If $\chr(F)=p$ and $a,b\in F$, then
\[ (a+b)^p=a^p+b^p.\]
\end{prop}
\begin{proof} 
We will use the binomial theorem, that
\[ (a+b)^n=\sum_{i=0}^n \binom{n}{i} a^ib^{n-i}\]
where the coefficients
\[ \binom{n}{i} = \frac{n(n-1)(n-2)\cdots(n-i+1)}{i!}.\]
It can be proven by induction in any commutative ring.  Moreover, one
can show that the coefficients $\binom m i$ are integers by induction.

In the case where $n=p$ and $1\leq i\leq p-1$, there is a factor of
$p$ in the numerator, but all factors in the denominator are
relatively prime to $p$ (they are less than $p$ since $p$ is prime).
So, the coefficient is a multiple of $p$ making that term $0$ in
characteristic $p$.
\end{proof}
With this in hand, we can define Frobenius maps.
\begin{prop}
  If $F$ is a field with characteristic $p$ prime, then
  \[ \frobp: F\to F\]
  given by $\frobp(a)=a^p$ is a ring homomorphism.
\end{prop}
\begin{proof}
  We need to check that $\frobp$ respects both addition and
  multiplication.  For all $a,b\in F$,
  \[ \frobp(ab)=(ab)^p=a^pb^p=\frobp(a)\frob(b)\]
  since multiplication is commutative, and
  \[  \frobp(a+b) = (a+b)^p = a^p+b^p=\frobp(a)+\frobp(b)\]
  where the key middle step follows from Prop.~\ref{dream}.
\end{proof}

Since a ring homomorphism is a group homomorphsim for addition, we get
\begin{cor}
  If $F$ is a field of characteristic $p$, then $\frobp:F\to F$ is
  injective.
\end{cor}
\begin{proof}
  By Proposition~\ref{invimgcount}, it suffices to show that the
  kernel is trivial.  But
\begin{align*}
  a\in \ker(\frobp) &\iff \frobp(a)=0 \\
  &\iff a^p=0 \\
  &\iff a=0
\end{align*}
since in a field, if a product of elements is zero, then one of the
factors must be zero.
\end{proof}
If we compose $\frobp$ with itself, we get the function
\[\frobp \of \frobp(a) =\frobp(\frobp(a)) =\frobp(a^p)=(a^p)^p=a^{p^2},\]
and similarly
\[\frobp \of \frobp(a)\of\frobp =\frobp(a^{p^2})
=(a^{p^2})^p=a^{p^3}.\]
By induction, we see that we compose $\frobp$ with itself $n$ times,
we get the function we denote $\frob_{p^n}(a)=a^{p^n}$.
Since the composition of two injective homomorphisms is an injective
homomorphism, we have that $\frob_{p^n}:F\to F$ is an injective ring
homomorphism for all $n\in\Z^+$.  To simplify notation, we let $q=p^n$
and write $\frobq$.

The field $\Fpbar$ is a field of characteristic $p$, so we can consider
$\frobq$ on $\Fpbar$ (where $q=p^n$).  The question is, what are its
fixed points.
\begin{prop}
  If $p$ is prime, $n\in\Z^+$, and $q=p^n$, then 
  \[ \{a\in \Fpbar\mid \frobq(a)=a\}=\F_{q}.\]
\end{prop}
The right-hand side of the equation is the unique field of order $q$.
\begin{proof}
  We first note that if $a\in \Fpbar$ and $\frobq(a)=a$, the $a^q=a$,
  i.e., $a$ is a root of $f(x)=x^q-x$.  This polynomial has exactly $q$
  roots in $\Fpbar$.  Note, $f'(x)=qx^{q-1}-1=-1$ ($q=p^n=0$ as a
  multiplier in characteristic $p$), so $f(x)$ and $f'(x)$ have no
  common roots.  Hence, there are exactly $q$ distinct roots of
  $x^q-x$ in $\Fpbar$.

  Now if $b\in \F_q$, we consider two cases.  If $b=0$, then
  $b^q-b=0^q-0$, so $b$ is a root of $x^q-x$.  If $b\neq 0$, then $b$
  is an element of the group $\F_q^\times$, which is a finite group of
  order $q-1$.  Hence $b^{q-1}=1$,  Multiplying by $b$, we get
  $b^q=b$, and so $b$ is a root of $x^q-x$.

  Comparing the results of the two paragraphs, the field $\F_q$
  provides $q$ fixed points for $\frobq$, and there are exactly $q$
  fixed points, so the two sets are equal.
\end{proof}
Note, the second half of this proof is often used to prove Fermat's
Little Theorem.  In abstract algebra, the material is usually arranged
a bit differently, and this argument is part of the proof of the
existence and uniqueness of the field $\F_q$.

\section{Group algorithms}

Here we present three algorithms which apply to any group.  Particular
instances of these may be familiar from other classes.

\subsection{Square and multiply}
The name for this algorithm applies when the group is written
multiplicatively.  Of course, it is useful in additive groups, like
the group of an elliptic curve.

Given a group $G$, an element $g\in G$, and $n\in\Z^+$, the goal is to
compute $g^n$ efficiently.  If we use the ``naive algorithm'' of
repeated multiplication, this takes $n-1$ steps.

There are two ideas involved.  For the first, suppose $n$ is a power
of $2$, such as $n=16=2^4$.  Then, we can compute $g^{16}$ with just
$4$ multiplications:
\begin{align}
g^2&= g\cdot g \label{sqrs}\\
g^4&= (g^2)\cdot (g^2) \notag\\
g^8&= (g^4)\cdot (g^4) \notag \\
g^{16}&= (g^8)\cdot (g^8) \notag
\end{align}
In short, this uses basic exponent rules: $(g^{2^j})^2=g^{2\cdot
  2^j}=g^{2^{j+1}}$.

Now if the exponent is not a power of $2$, we write it as a sum of
distinct powers of $2$.  This is always possible, and is equivalent to
writting the number in base $2$.  We compute the powers as above, and
then multiply together what we need.
\begin{exam}
  Suppose we want to compute $g^{19}$.  We first write
  \[ 19=2^0+2^1+2^4\]
  One way to find this is to start with the largest power of $2$ which
  is $\leq 19$, subtract it from $19$, and repeat.

  Next we compute squarings exactly as in equation~\eqref{sqrs} (four
  steps), saving all of the results.  Then we compute
  \[ g^{19} = g^{2^0+2^1+2^4} = g^{2^0}\cdot g^{2^1}\cdot g^{2^4}
  = g\cdot g^2\cdot g^{16}\]
and each of these three values have already been computed.  There are
two more multiplications needed in the final expression.  The result
is computed with $4$ squarings and $2$ multiplications, so $6$
operations as oposed to $18$ with the naive algorithm.
\end{exam}

Historically, this is one of the oldest recorded algorithms when
applied to the problem of multiplying two positive integers, which can
be viewed as ``repeated addition'' in the additive group $\Z$, and is
also known as Russian Peasant Multiplication.

\subsection{Order of an element with a multiplicative bound}
Let $G$ be a group and $g\in G$.  The question is: if we know $|g|$ is
a divisor of $m$ for some $m\in\Z^+$, then how do we efficiently find
$|g|$.  We give an algorithm which runs in polynomial time provided we
can somehow factor $m$.  

The algorithm is based on the following proposition.

\begin{prop}
  Suppose $G$ is a group, $g\in G$, and $m\in\Z^+$, such that $g^m=e$,
  and $p$ a prime dividing $m$.  If $g^{m/p}\neq e$, then the exponent
  of $p$ in the factorization of $m$ equals the exponent of $p$ in the
  factorization of $|g|$.
\end{prop}
\begin{proof}
  Let $m=p_1^{a_1}\cdots p_k^{a_k}$ be the prime power factorization
  of $m$.  Reordering the factors, we can assume $p=p_1$.  Let
  $n=|g|$.  Since $g^m=e$, $n\mid m$, and so we can factor $n$ over
  the same set of primes $n=p_1^{b_1}\cdots p_k^{b_k}$ with $0\leq
  b_i\leq a_i$ for all $i$.  We want to show $b_1=a_1$.

  So, suppose $b_1<a_1$.  Then $n\mid (m/p)$ (recall $p=p_1$), which
  implies $g^{m/p}=e$ \con.  Thus, $b_1=a_1$.
\end{proof}

If we apply the proposition to every prime dividing $m$, we get the
following corollary which gives a criterion for checking if a value
$m$ is the order of $g$.
\begin{cor} \label{order-check}
  Suppose $G$ is a group, $g\in G$, and $m\in\Z^+$ such that
  \begin{itemize}
  \item $g^m=e$
  \item $g^{m/p}\neq e$ for every prime $p\mid m$
  \end{itemize}
  Them $m$ is the order of $g$.
\end{cor}

For the algorithm, suppose $G$ is a group, $g\in G$, $m\in\Z^+$ and
$g^m=e$.  We then
\begin{enumerate}
\item Factor $m=p_1^{a_1}\cdots p_k^{a_k}$.
\item Loop over the primes $p_i$.  For each $i$,
  \begin{enumerate}
  \item If $g^{m/p_i}=e$, replace $a_i$ with $a_i-1$.  If now $a_i=0$,
    move to the next prime.  If $a_i$ is still positive, repeat this
    step with the same prime $p_i$.
    \item If $g^{m/p_i}\neq e$, move to the next prime $p_i$.
  \end{enumerate}
\end{enumerate}

\begin{exam}
 Let  $p=7129$, which is prime.  We want to find the order of $3467$
 modulo $p$, i.e., in $\Z_{7129}^\times$.  By Fermat's Little Theorem,
 we know $3467^{7128}\equiv 1\pmod{7129}$, so we have a starting value
 of $m$.  We then factor $7128=2^3 3^4 11$, so we have three primes to
 check.
\begin{align*}
  3467^{7128/2}&\equiv 7128\not\equiv 1\pmod{7129} \\
    3467^{7128/3}&\equiv 1\pmod{7129}
\end{align*}
So, we revise of value of $m$ to $(2^33^4 11)/3=2376$ and check
\begin{align*}
    3467^{2376/3}&\equiv 1\pmod{7129}
\end{align*}
So, we revise again to $m=(2^3 3^3 11)/3=792$ and check
\begin{align*}
    3476^{792/3}&\equiv 5879\not\equiv 1\pmod{7129} \\
    3476^{792/11}&\equiv 5899\not\equiv 1\pmod{7129}
\end{align*}
So, the order of $3476$ in $\Z_{7129}^\times$ is $729$.  Note, we did
not have to recheck the exponent on $2$.
\end{exam}
There are lots of powers to compute in the example.  They can be
computed by square and multiply, and even that can be sped up since we
are always computing powers of $3476$ modulo $7129$.  So, we would
only need to do the series of squarings once.  Then for each power, we
just have to do the ``multiply'' stage.

\subsubsection{Primality proving}
For some cryptosystems, it is important to have large prime numbers.
A test like Rabin-Miller is effective in finding composite numbers,
but does not prove that an input is prime.
Corollary~\ref{order-check} is the basis of a primality proving
algorithm called the Pocklington-Lehmer test.  The algorithm we
present here is a slight simplification of the test, but the
difference in practice is negligible.

Suppose we have $m\in\Z^+$ we think is prime and want to prove it is
prime.  If we can find an integer $a$ which has order $m-1$, then
$\varphi(m)=m-1$ since
\[ m-1 \geq \varphi(m)=|\Z_m^\times|\geq |a| = m-1,\]
so we have equality all the way across.  But, $\varphi(m)=m-1$ means
all of the numbers $1,2,\ldots, m-1$ are relatively prime to $m$.  In
particular, $m$ has no non-trivial divisors, so $m$ is prime.

So, how can we find this value of $a$, just try $a=2, 3, 4, \ldots$
until we find a winner.  If $m$ really is prime, such values must
exist (existence of primitive roots modulo $p$), and their proportion
cannot be too small.  For each $a$, we apply
Corollary~\ref{order-check} to see if it has order $m-1$.

The slowest part of this primality proving algorithm is that we must
factor $m-1$.  So, it is not useful in all cases, but can be used to
prove primality of numbers of special forms (where $m-1$ is easy to
factor).


\subsection{Order of an element given an archimedian bound}

Let $G$ be a group and $g\in G$.  The question is: if we know
$|g|\leq B$ for some bound $B$, then how do we efficiently find
$|g|$. Algorithms for this are given in the text.  Baby-step
Giant-step is a deterministic algorithm (with $O(\sqrt{B})$ steps and
$O(\sqrt{B})$ storage), and Pollard's rho is a probabilistic algorithm
(with $O(\sqrt{B})$ steps and constant storage).

\section{Number theory background}

We start with a definition.
\begin{defn}\label{divides}
  If $a,b\in\Z$, then we say that $a$ \term{divides} $b$
  and write $a\mid b$ if there exists an integer $c$ such that $ac=b$.
\end{defn}

\begin{thm}[Division algorithm] \label{div-alg}
  If $a\in \Z$ and $b\in\Z^+$, then there exists a unique pair of integers $q$ and $r$
  such that $a=bq+r$ and $0\leq r<b$.
\end{thm}
With $a,b,q,r$ as above, we write $r=a\bmod b$.  Note, $r=0$ if and
only if $b\mid a$.

\begin{defn}
  If $a,b\in\Z$, then
  \begin{itemize}
  \item if $a=b=0$, then we define $\gcd(a,b)=0$,
  \item if $a\neq 0$ or $b\neq 0$, then we define $\gcd(a,b)$ to be
    the largest integer $d$ such that $d\mid a$ and $d\mid b$.
 \end{itemize}
 \end{defn}

The next theorem is also refered to as the Extended Euclidean
Algorithm since values for $r$ and $s$ can be computed from the
Euclidean algorithm plus a little more work.

\begin{thm}[Bezout property] \label{bezout}
  If $a,b\in\Z$, then there exist integers $r$ and $s$ such that
  $$ ra+sb=\gcd(a,b).$$
\end{thm}

The next proposition is often proven as a step towards proving unique
factorization in $\Z$.  The name comes from ring theory where one
distinguishes prime elements from irreducible elements of a ring (the
point here is that they are the same in $\Z$).

\begin{prop}[Prime property] \label{prime-prop}
  Let $n\in\Z$, $n>1$.  Then $n$ is prime if and only if
  \[ \forall a,b\in\Z, n\mid ab\implies n\mid a \text{ or } n\mid b.\]
\end{prop}
\begin{proof}
  $(\Rightarrow)$ Suppose $n\nmid a$.  Since $\gcd(a,n)$ divides $n$
  and $n$ is prime $\gcd(a,n)=1$ or $n$.  But $n\nmid a$, so
  $\gcd(n,a)=1$.  Then by the Bezout property (Thm.~\ref{bezout}), there
  exists $r,s\in\Z$ such that $ra+sn=1$.  Multiplying by $b$ we get
  $r(ab) + snb = b$.  Both terms on the left side are multiples of $n$,
  so $n$ divides the right side, i.e., $n\mid b$.

$(\Leftarrow)$
Suppose not.  Since $n>1$ and is not prime, $n=ab$ for some $1<a\leq
b<n$.  Thus $n\mid ab$, and so $n\mid a$ or $n\mid b$, which
contradicts the inequalities on $a$ and $b$ \con.
\end{proof}

\begin{prop}
  If $n\in\Z^+$, $a,b\in\Z$ such that $\gcd(a,n)=\gcd(b,n)=1$, then
  $\gcd(ab,n)=1$.
\end{prop}
\begin{proof}
  Suppose not.  Then there exists some prime $p$ such that $p\mid
  \gcd(ab,n)$.  But then $p\mid ab$ and $p\mid n$.  By the Prime
  Property (Prop.~\ref{prime-prop}), $p\mid a$ or $p\mid b$.  Thus, $1<p\leq
  \gcd(a,n)$ or $1<p\leq \gcd(b,n)$ \con.
\end{proof}

\section{Crypto 101}
\label{crypto-101}
Here we provide the basic set up and notation we use for basic
cryptography.  We have a set of possible messages, $\calP$, called the
plaintexts.  It might be the set of $26$ letters, or the set of all
$4$-letter strings, or anything.  A cryptosystem has a set of keys
$\calK$, and a set of encoded  texts $\calC$ called ciphertexts.  The
cryptosystem provides encryption and decryption functions depending on
the key being used.  That is, for each $K\in\calK$, we have functions
\[ e_K:\calP\to \calC \quad \text{ and }\quad d_K:\calC\to \calP\]
so that encryption followed by decryption returns the original
message:
\[ d_K\circ e_K = I_{\calP}\]
(the right hand side denotes the identity function on the set of
plaintexts).

Many cryptosystems encode one letter at a time, so $\calP$ may be the
set of single letters (e.g., shift and affine ciphers), but others
work naturally on blocks of several letters at a time (e.g., the Hill
and Vigen\`ere ciphers).  In those cases, we let $\calA$ be the set of
letters (i.e., our alphabet), and generally $\calP=\calC=\calA^n$, the
set of $n$-letter strings for some fixed $n\in\Z^+$.

\section{Lagrange's theorem}
We make frequent use of Lagrange's theorem from group theory.
\begin{thm}[Lagrange's theorem]
  If $G$ is a finite group and $H$ is a subgroup of $G$, then $|G|$ is
  a multiple of $H$.
\end{thm}
Most of the ingredients are already established.  In
Section~\ref{coset-home}, we defined cosets of a subgroup and proved
that all cosets of $H$ have $|H|$ elements.  To prove Lagrange's
theorem, we need just one more lemma.
\begin{lemma}
  If $H$ is a subgroup of a group $G$, then the left cosets of $H$
  partition $G$.
\end{lemma}
\begin{proof}
  For all $g\in G$, $g=ge\in gH$, so every element of $G$ is in at
  least one coset.  What remains is to prove that if two cosets have
  non-trivial intersection, then they are the same set.

  Suppose $a,b\in G$ and $aH\cap bH\neq \emptyset$.  Now, there exists
  $x\in aH\cap bH$.  
  So, there exists $h,k\in H$
  such that $x=ah$ and $x=bk$.  Thus $ah=bk$, which implies
  $a=bkh^{-1}$.  

  Let $y\in aH$.  Then $y=at$ for some $t\in H$.
  Substituting, we get
  \[ y = at = (bkh^{-1}) t = b(kh^{-1} t)\in bH.\]
  The final step follows because $H$ is a subgroup, and $h,k,t\in H$.
  Thus, $aH\seq bH$.  The reverse inclusion now follows by reversing
  the roles of $a$ and $b$, so $aH=bH$. 
\end{proof}
\begin{proof}[Proof of Lagrange's theorem]
  We count the elements of $G$ by counting them by coset.  Each coset
  has the same number of elements and they partition $G$, so 
  \[ |G| = |H|\cdot \text{ the number of cosets of $H$.}\]
  Therefore, $|G|$ is a multiple of $|H|$.
\end{proof}
To see how this works, it helps to see some examples of cosets.
\begin{exam}
  Let $G=\Z_7^*$, and $H=\{1,6\}$.  To find the left coset of say $2$,
  we just compute
  \[ 2H = \{2\cdot 1, 2\cdot 6\} = \{2, 5\}\]
  where we do the computation modulo $7$.  Now all of the cosets are
  \begin{align*}
    1H &= \{1, 6\} \\
    2H &= \{2, 5\} \\
    3H &= \{3, 4\} \\
    4H &= \{4, 3\} \\
    5H &= \{5, 5\} \\
    6H &= \{6, 1\} 
  \end{align*}
  We have $6$ elements in all, and each generates a left coset, but
  there is duplication: $1H=6H$, $2H=5H$, and $3H=4H$.  So, there are
  only $3$ distinct left cosets, they are pairwise disjoint, and each
  has $2$ elements.  The count for Lagrange's theorem is then $3$
  cosets with $2$ elements each for a total of $3\cdot 2=6$ elements.
\end{exam}
\begin{prob}
  Repeat the computation in the example using $G=\Z_7^*$ and
  $H=\{1,2,4\}$.
\end{prob}


\section{LFSRs} \label{lfsr}
Suppose we have a plaintext which has been
converted to binary form, i.e., a sequence of $0$s and $1$s.  The
optimal encryption would be a {\em one-time pad}, where a random
binary sequence is generated, and xor'ed (i.e., added modulo $2$) to
the plaintext to produce the cipher text. This encryption has {\em
  perfect secrecy} -- looking at the ciphertext, the probability that
it came from any given plaintext is the same as it was without knowing
the ciphertext.  In essence, the ciphertext adds no information.

One-time pads are generally impractical, in part because the
encryption key is as long as the plaintext and must also be
communicated to the reader.  In a stream cipher, one
generates a long binary sequence which looks random, but which can be
generated from a much smaller amount of information.

\term{Linear feedback shift registers}, or LFSRs, are stream ciphers
which can be implemented in hardware and which run extremely quickly,
so they can be used to encrypt large amounts of data.  One of the
historical uses was in military field telephones.  A small amount of
key information was input to the phone, and then it would be able to
encrypt/decrypt audio in real time.

Let $n\in\Z^+$.  In an $n$-stage LFSR, we have $n$ memory locations
which each holds a bit, i.e., a $0$ or a $1$.  We visualize these as
$n$ boxes in a row.  When we run the LFSR by one step forward, the
bits shift one box to the left.  The bit in the leftmost box comes
out, and is the next bit of the key stream.  The only thing remaining
is to fill the rightmost box.  For that, we use a fixed linear
combination of the bits in the registers taken modulo $2$.

\begin{comment}
Let $n$ be a positive integer.  An $n$-stage LFSR generates its key
stream by a linear recurrence relation
\[ x_{k+n} = c_{n-1} x_{k+n-1}\pots c_0 x_k\]
where $(c_0\cots c_{n-1})\in\Z_2^n$ is a fixed vector called the
taps.  Note that if $c_0=0$, then the recurrence relation has
effectively at most $n-1$ terms, so would be an $n-1$-stage LFSR.  So,
we assume $c_0=1$.
\end{comment}

We can picture an $4$-stage LFSR as follows

  \xymatrix@R-5pt{
& \ar[rrrr] & &  &  &  {\bigoplus} \ar `d/1pt[d] [ld] \\
\ldots01011001
&   *+[F]{x_n} \ar[l] \ar[u]_{c_1} & *+[F]{x_{n+1}} \ar[l]
  \ar[u]_{c_2} &*+[F]{x_{n+2}} \ar[l] \ar[u]_{c_3} &*+[F]{x_{n+3}} \ar[l]
  \ar[u]_{c_4}&   \\
&
}

Here, sequence of bits which is output by the LFSR is denoted $x_1$,
$x_2$, $x_3$, \ldots.  The linear combination used to fill the
rightmost box is
\begin{align}
x_{n+4} &= c_4x_{n+3}+c_3 x_{n+2}+c_2 x_{n+1}+ c_1x_n \quad \bmod 2 \label{lfsrrecur}\\
&= \sum_{j=1}^4 c_j x_{n+j-1}\quad \bmod 2
\end{align}
The values $c_1, \ldots, c_4\in\{0,1\}$ are called the taps.  To get
things started, we also need the initial contents of the boxes $(x_1,\ldots, x_4)$.

LFSRs have been used in cases where one needs to encrypt large amounts
of data quickly, such as digitized audio in military phones used in
the field.  The bit operations can be implemented fairly easily in
hardware, making them fast.

Mathematically, we can view this as simply a sequence defined by a
linear recursion, namely equation~\eqref{lfsrrecur}.  In general, a
linear recursive sequence can be thought of as coming from matrix
multiplication:
\[ C 
\begin{pmatrix} x_i \\ x_{i+1} \\ \vdots \\ x_{i+n-1} \end{pmatrix} =
\begin{pmatrix} x_{i+1} \\ x_{i+2} \\ \vdots \\ x_{i+n} \end{pmatrix}
\]
where (for a $4$-stage LFSR)
\[ C = \begin{pmatrix} 0 & 0 & 0 & -c_1 \\
1 & 0  & 0 & -c_2 \\
0 & 1  & 0 & -c_3 \\
0 & 0 & 1 & -c_4 
\end{pmatrix}
\]
Shifting the LFSR one time corresponds to multiplying by $C$ once,
running it $2$-steps corresponds to multiplying by $C$ twice, or
equivalently by $C^2$, and so on.  In fact,
\[ C^{j-1}
\begin{pmatrix} x_1 \\ x_{2} \\ \vdots \\ x_{n} \end{pmatrix} =
\begin{pmatrix} x_{j} \\ x_{j+1} \\ \vdots \\ x_{j+n=1} \end{pmatrix}
\]
So, understanding the behavior of the LFSR comes by understanding
powers of $C$.

A standard convention for LFSRs is to assume $c_1=1$; otherwise, the
exact same stream of bits could be generated by a shorter LFSR.  With
this assumption, the matrix $C$ above has non-zero determinant (modulo
$2$), so is invertible.  In other words, $C\in \GL_n(\Z_2)$, and we
would need to analyze the powers of this group element.

Some of the basic results on LFSRs are as follows.  Here we fix the
taps, or equivalently, fix the matrix $C$.  By the state of the LFSR
we mean the contents of the $n$ memory locations.
\begin{enumerate}
  \item From any time forward, all future states are determined by the
    current loading of the LFSR.
  \item The output stream is ultimately periodic with period of at
    most $2^n$ (by the Pidgeon hold principle, there has to be a
    repeat in the state of the LFSR after at most $2^n$ steps).
  \item With the convention $c_1=1$, the process is reversible (the
    matrix $C$ is invertible).  So, the output is a purely periodic
    sequence.
  \item If the state has all $0$s, then it remains that way going
    forward.  Similarly, if it ever reaches a state of all $0$s, then
    the previous, hence all previous, state was all $0$s.
  \item So, if the initial load is not all $0$s, then output is a
    purely periodic sequence with period of at most $2^n-1$.
\end{enumerate}
Using linear algebra to analyze this further, one can prove that for
each $n$, there is a set of taps so that one can achieve a period of $2^n-1$.




\begin{comment}
\newpage
\section{Rest}

\begin{itemize}
\item Examples: additive groups
\item Examples: multiplicative groups
\item Integers mod $n$
\begin{itemize}
\item Connect congruences to congruence classes to working with
  remainders
\item Both additive and multiplicative groups
\end{itemize}
\item Cyclic subgroups
\item Order of an element and basic facts about powers
\item Connect to shift cipher
\item Related to RSA
\item Connected to $p-1$ factorization
\item $GL_n(\Zn)$ and Hill cipher
\item Products of groups
\item Isomorphisms ($\R$ and $\R^+$)
\item Structure of finite abelian groups -- compare with text
\end{itemize}


\section{Elliptic curves}
\begin{itemize}
\item Cannonball problem, square base, total is a square:
\[ \sum_{j=1}^n j^2 = k^2\]
or
\[ \frac{j(j+1)(2j+1)}{6} = k^2\]

\item Fermat's last theorem, $n=3$ case

\item Congruent number problem: which integers $n$ equal the area of a
  right triangle with rational sides is connected

\item Fermat's last theorem, hard case $n=p$ with $p$ large: if
\[ a^p+b^p=c^p\]
consider $y^2=x(x-a^p)(x+b^p)$.

\item Define, but then move to typical coordinates

\item Simplify coords

\item $(\prod_{i<j} r_i-r_j)^2=4A^3+27B^2\neq 0$

\item General equation

\item Group law:
\[ m=\frac{y_2-y_1}{x_2-x_1}\quad y=m(x-x_1)+y_1\]
Substitute
\[ 0=x^3-m^2x^2+\cdots\]
Divide away two roots, or note $x^3+ax^2+bx+c = (x-r)(x-s)(x-t)$ has
$a=-(r+s+t)$, so $x_3 = -a-x_1-x_2 = m^2-x_1-x_2$ and
$y=m(x-x_1)+y_1$.
\item Then reflect: $x_3= m^2-x_1-x_2$, $y_3=m(x_1-x_3)-y_3$

\item $P_1=P_2$, $2yy'=3x^2+A$, so $m=(3x_1^2+A)/(2y_1)$, continue as
  before to get $x_3=m^2-2x_1$, $y_3 = m(x_1-x_3)-y_1$ (negated)

\item This gives an abelian group

\item $y^2=x^3+x+1$ over $\Z_5$.  Naive point search gives
  $$\{(2,1),(2,4),(3,1),(3,4), (4,2), (4,3), (0,1), (0,4),
  \mathcal{O}\}$$
  a group with 9 points.

  Doubling: $m=(3x^2+1)*3/y$, so for first point, we get for first
  point $m=4$, $x=1-2\cdot 2=2$, $y=4(2-2)-1=4$


\item Talk about $\Q$, $\R$, $\C$, and finite fields ($S^1$ for circle
  group)
\end{itemize}
\end{comment}

\end{document}

%%% Local Variables: 
%%% mode: latex
%%% TeX-master: t
%%% End: 
